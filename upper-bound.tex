\section{Upper bound for the limit set}
\label{sec:upper-bound-limit-set}

In this section, we prove that $\geolim(\mcg(\no_g))$ is contained in $\pmf^+(\no_g)$.
We do so by defining an $\mcg(\no_g)$-invariant subset $\systole(\no_g)$, and showing that the intersection of its closure with $\pmf(\no_g)$ is contained in $\pmf^+(\no_g)$.

\begin{definition}[One-sided systole superlevel set]
  For any $\varepsilon > 0$, the set $\systole(\no_g)$ is the set of all points in $\teich(\no_g)$ where the length of the shortest one-sided curve is greater than or equal to $\varepsilon$.
\end{definition}

We can state the main theorem of this section.
\begin{theorem}
  \label{thm:systole-closure}
  For any $\varepsilon > 0$, $\overline{\systole(\no_g)} \cap \pmf(\no_g)$ is contained in $\pmf^+(\no_g)$.
\end{theorem}
The key idea of the proof is proving a quantitative estimate on the Fenchel-Nielsen coordinates of points converging to points in $\pmf^-(\no_g)$.
\begin{proposition}
  \label{prop:pinching}
  Let $\{m_i\}$ be a sequence of points in $\teich(\no_g)$ converging to a projective measured foliation $[\lambda]$.
  If $p$ is a one-sided atom of $\lambda$, for any Fenchel-Nielsen coordinate chart containing $p$ as a cuff, the length coordinate of $p$ goes to $0$.
\end{proposition}

\textit{Outline of proof.} The proof of \autoref{prop:pinching} proceeds in two steps:
\begin{enumerate}[(i)]
\item We first show that there is a curve $p_3$ intersecting $p$ such that $p_3$ is left invariant by Dehn twisting along the two-sided curve that deformation retracts onto $2p$ (when $p$ is thought of as an element in $\pi_1(\no_g)$).
  We do so in  \autoref{claim:one-sided-rigidity} and \autoref{claim:intersects-atmost-once}.
  This gives an upper bound for the length of $p_3$ in terms of the length of $p$, and an orthogeodesic going through $p$.
\item We use the upper bound obtained in the previous step to show that if the length of $p_3$ goes to $\infty$, the length of $p$ must go to $0$.
  This result can be thought of as a converse to the collar lemma, using the additional hypotheses we manage to obtain from the previous step.
\end{enumerate}

\begin{proof}[Proof of \autoref{prop:pinching}]
  Consider the following decomposition of the measured foliation $\lambda$.
  \begin{align*}
    \lambda = 1 \cdot p + \lambda_{\mathrm{at}} + \lambda_{\mathrm{Leb}}
  \end{align*}
  Here, $\lambda_{\mathrm{at}}$ are the minimal components on periodic components other than $p$, i.e. cylinders and Möbius strips, and $\lambda_{\mathrm{Leb}}$ are non-periodic minimal components.
  In the above expression, $p$ is the one-sided curve considered as a measured foliation (since we're picking a representative of $[\lambda]$, we can pick one such that $p$ has weight $1$).

  Pick simple closed curves $p_0$, $p_1$, and $p_2$, where $p_0$ is the curve $p$, and $\{p_0, p_1, p_2\}$ bound a pair of pants.
  Furthermore, we impose the following conditions on $p_1$ and $p_2$.
  \begin{align*}
    i(p_1, \lambda_{\mathrm{at}}) &= 0 \\
    i(p_2, \lambda_{\mathrm{at}}) &= 0
  \end{align*}
  Note that this can always be done, by deleting the support of $\lambda_{\mathrm{at}}$, and looking at the resulting subsurfaces.
  Neither $p_1$ nor $p_2$ can be the same as $p_0$, since $p_0$ is one-sided.

  Consider now a collection of curves $\{q\}$ which satisfy the following two constraints.
  \begin{enumerate}[(i)]
  \item $i(q, p_0) = 1$.
  \item $i(q, p_1) = 0$ and $i(q, p_2) = 0$.
  \end{enumerate}
  We use the fact that $p_0$ is one-sided to make the following claim.
  \begin{claim}
    \label{claim:one-sided-rigidity}
    There is exactly one curve $q$ up to homotopy that satisfies conditions $(i)$ and $(ii)$.
  \end{claim}
  \begin{proof}
    Let $q_1$ and $q_2$ be two curves satisfying both the conditions.
    We can assume without loss of generality that both $q_1$ and $q_2$ intersect $p_0$ at the same point.
    We now delete the curves $p_0$, $p_1$, and $p_2$ to get a pair of pants $\mathcal{P}$: denote the boundary component corresponding to $p_0$ by $\widetilde{p_0}$, and the arcs corresponding to $q_1$ and $q_2$ by $\widetilde{q_1}$ and $\widetilde{q_2}$.
    Since $p_0$ was one-sided, $\widetilde{q_1}$ and $\widetilde{q_2}$ intersect $\widetilde{p_0}$ at two points, which are diametrically opposite (with respect to the induced metric on the geodesic $\widetilde{p_0}$).

    On a pair of pants, two arcs going from a boundary component to the same component must differ by Dehn twists along that component up to homotopy relative to the boundary components: this is a consequence of the fact that the mapping class group of $\mathcal{P}$ is $\mathbb{Z}^3$, where each $\mathbb{Z}$ component is generated by a Dehn twist along a boundary component.
    This means that there is a some Dehn twist $D$ along the boundary component $\widetilde{p_0}$ such that $D\widetilde{q_1}$ is homotopic to $\widetilde{q_2}$ relative to its endpoints. Let $\widetilde{q_2}$ now denote $D\widetilde{q_1}$.

    We claim that after quotienting $\widetilde{p_0}$ by the antipodal map, $\widetilde{q_1}$ and $\widetilde{q_2}$ map to homotopic curves.
    The homotopy is obtained by moving the point of intersection of $\wt{q_2}$ and $p_0$ twice around the curve $p_0$.

    \autoref{fig:q1q2} shows the two arcs on $\mathcal{P}$ and \autoref{fig:q1toq2} shows the homotopy on the quotient that takes $\widetilde{q_2}$ to $\widetilde{q_1}$ (the movement of the blue arc is indicated by the blue arrows in \autoref{fig:q1toq2}).
    \begin{figure}[h]
      \centering
      % \includegraphics{example-image-a}
      \incfig[0.6]{q1q2}
      \caption{The arcs $\widetilde{q_1}$ and $\widetilde{q_2}$.}
      \label{fig:q1q2}
    \end{figure}

    \begin{figure}[h]
      \centering
      % \includegraphics{example-image-b}
      \incfig[0.5]{q1toq2}
      \caption{Homotopy taking $q_2$ to $q_1$.}
      \label{fig:q1toq2}
    \end{figure}

    We have thus constructed the desired homotopy from $q_1$ to $q_2$.
    The example in \autoref{fig:q1q2} also shows there is at least one such curve, proving the claim.
  \end{proof}

  Let $p_3$ be the geodesic representative of the curve described in Claim \ref{claim:one-sided-rigidity}.
  We also define $p_4$ to be the orthogeodesic arc from $p_0$ to itself.
  We make the following claim about $p_3$ and $p_4$.

  \begin{claim}
    \label{claim:intersects-atmost-once}
    The arc $p_4$ and the curve $p_3$ intersect at most once.
  \end{claim}
  \begin{proof}
    We know from Claim \ref{claim:one-sided-rigidity} that $p_3$ is homotopic to any other curve which intersects $p_0$ exactly once and does not intersect $p_1$ and $p_2$.
    It then suffices to construct a curve $q$ that intersects $p_4$ at most once: since $p_3$ is the geodesic representative of $q$, it will also intersect $p_4$ at most once.
    We construct $q$ by starting along $p_0$, near the point where $p_4$ intersects $p_0$, and then travel parallel to $p_4$.
    When the curve reaches $p_0$ again, it will need to turn left or right to close up. In one of these cases, it will have to intersect $p_4$ once, and in the other case, it will not intersect $p_4$ at all.
  \end{proof}
  With claims \ref{claim:one-sided-rigidity} and \ref{claim:intersects-atmost-once}, we have the following picture of $\left\{ p_0, p_1, p_2, p_3, p_4 \right\}$ on the pair of pants.
  \begin{figure}[h]
    \centering
    \incfig[0.75]{pants}
    \caption{The curves restricted to a pair of pants.}
    \label{fig:pants}
  \end{figure}


  Since $i(p_3, p_0) = 1$, and $p_0$ is a component of the limiting foliation, the length of $p_3$ must go to $\infty$.
  On the other hand, we can bound the length of $p_3$ above and below via the lengths of the orthogeodesic $p_4$ and the length of $p_0$.
 \begin{equation}
   \label{eq:up-1}
   \ell(p_3) \leq \ell(p_4) + \ell(p_0)
 \end{equation}
 Observe that the upper bound follows from \autoref{claim:intersects-atmost-once} and the fact that the red and cyan arcs are isotopic to $p_3$ relative to their endpoints being fixed.
 The cyan arcs have length at most $\ell(p_0)$ in this setting; if one allowed a twist parameter, the length of the cyan arcs would be proportional to the twist parameters.
 The point of this inequality is that we can estimate $\ell(p_4)$ using $\ell(p_0)$, $\ell(p_1)$ and $\ell(p_2)$ via hyperbolic trigonometry.
 Cut the pair of pants along the seams, to get a hyperbolic right-angled hexagon, pictured in \autoref{fig:hexagon}.
 \begin{figure}[h]
   \centering
   \incfig[1]{hexagon}
   \caption{The right angled hexagon obtained by cutting the pants along the seams.}
   \label{fig:hexagon}
 \end{figure}

 To get good estimates on $\ell(p_4)$, we need a universal lower bound on the fraction $f$ as we move in the Teichm\"uller space.
 The analysis splits up into two cases, but it is not {a priori} clear that these two cases are exhaustive.
 We will deal with the two cases, and then show that any other case can be reduced to the second case by changing $p_1$ and $p_2$.

\subsection*{Case I}
\label{case1}
We're in this case if $p_1$ and $p_2$ don't intersect the foliation $\lambda$ at all.
\begin{align*}
  i(p_1, \lambda) &= 0 \\
  i(p_2, \lambda) &= 0
\end{align*}
In this case, we can pass to a subsequence of $\{m_i\}$ such that the corresponding values of $f$ are always greater than $\frac{1}{2}$ or less than $\frac{1}{2}$.
In the former case, we focus on $p_1$, and in the latter case, we focus on $p_2$.
Without loss of generality, we'll suppose $f \geq \frac{1}{2}$.
In that case, we cut along the orthogeodesic $p_4$, and get a hyperbolic right-angled pentagon, which is the top half of \autoref{fig:hexagon}.

Let $\ell_i(p_k)$ denote the length of $p_k$ on the hyperbolic surface corresponding to $m_i$.
We can relate $\ell_i(p_0)$, $\ell_i(p_1)$, and $\ell_i(p_4)$ using the following identity for hyperbolic right-angled pentagons (see \cite{thurston1979geometry} for the proof of the identity).
\begin{align}
  \label{eq:pentagon}
  \sinh\left( f \cdot \ell_i(p_0) \right)  \cdot \sinh\left( \frac{\ell_i(p_4)}{2} \right)  = \cosh\left( \frac{\ell_i(p_1)}{2} \right)
\end{align}
Now suppose for contradiction's sake that $\ell_i(p_0)$ does not go to $0$.
Then we must have that for all $i$, $\ell_i(p_0) \geq 2\varepsilon$ for some $\varepsilon > 0$. By the lower bound on $f$, we have that the first term on the left hand side of the above expression is bounded below by $\varepsilon$.
Rearranging the terms gives us the following upper bound on $\ell_i(p_4)$.
\begin{equation}
  \label{eq:up-2}
  \ell_i(p_4) \leq 2 \cdot \sinh^{-1} \left( \frac{\cosh \left( \frac{\ell_i(p_1)}{2} \right)}{\varepsilon} \right)
\end{equation}
Using \eqref{eq:up-1} and \eqref{eq:up-2}, we get an upper bound for $\ell_i(p_3)$.
\begin{equation}
  \label{eq:up-3}
  \ell_i(p_3) \leq \ell_i(p_0) +
  2 \sinh^{-1} \left( \frac{\cosh \left( \frac{\ell_i(p_1)}{2} \right)}{\varepsilon} \right)
\end{equation}

Since $\frac{i(p_0, \lambda)}{i(p_3, \lambda)} = 0$, as $\{m_i\}$ approaches $\lambda$, the ratio of lengths of $p_0$ and $p_3$ approach $0$.
\begin{equation}
  \label{eq:up-4}
  \lim_{i \to \infty} \frac{\ell_i(p_0)}{\ell_i(p_3)} = 0
\end{equation}
Using \eqref{eq:up-3}, we have the lower bound for $\frac{\ell_i(p_0)}{\ell_i(p_3)}$.
\begin{equation}
  \label{eq:up-5}
  \frac{\ell_i(p_0)}{  \ell_i(p_0) +
  2 \sinh^{-1} \left( \frac{\cosh \left( \frac{\ell_i(p_1)}{2} \right)}{\varepsilon} \right)} \leq \frac{\ell_i(p_0)}{\ell_i(p_3)}
\end{equation}
By \eqref{eq:up-4}, the left hand side of \eqref{eq:up-5} must go to $0$, or equivalently, the following holds.
\begin{align}
  \label{eq:up-5dot5}
\lim_{i \to \infty} \frac{\ell_i(p_0)}{2 \sinh^{-1} \left( \frac{\cosh \left( \frac{\ell_i(p_1)}{2} \right)}{\varepsilon} \right)}
= 0
\end{align}
But we also have that $\ell_i(p_0) > \varepsilon$: that means the only way that the above limit is $0$ if $\ell_i(p_1)$ goes to $\infty$.
This is where the hypotheses of the Case I come in.
Since $i(p_1, \lambda)$ is $0$, the following equality must hold.
\begin{equation}
  \label{eq:up-6}
  \lim_{i \to \infty} \frac{\ell_i(p_1)}{\ell_i(p_3)} = 0
\end{equation}
This means the lower bound for $\frac{\ell_i(p_1)}{\ell_i(p_3)}$ must go to $0$.
\begin{equation}
  \label{eq:up-7}
   \lim_{i \to \infty} \frac{\ell_i(p_1)}{  \ell_i(p_0) +
  2 \sinh^{-1} \left( \frac{\cosh \left( \frac{\ell_i(p_1)}{2} \right)}{\varepsilon} \right)
  } = 0
\end{equation}
From \eqref{eq:up-5dot5}, we have the following.
\begin{equation}
  \label{eq:up-7dot5}
  \lim_{i \to \infty} \frac{\ell_i(p_1)}{  \ell_i(p_0) +
  2 \sinh^{-1} \left( \frac{\cosh \left( \frac{\ell_i(p_1)}{2} \right)}{\varepsilon} \right)
  }
  =
  \lim_{i \to \infty} \frac{\ell_i(p_1)}{
  2 \sinh^{-1} \left( \frac{\cosh \left( \frac{\ell_i(p_1)}{2} \right)}{\varepsilon} \right)
  }
\end{equation}

But as $\ell_i(p_1)$ approaches $\infty$, the right hand side of \eqref{eq:up-7dot5} approaches a non-zero constant value, which contradicts the identity in \eqref{eq:up-7}.
This contradiction means our assumption that $\ell_i(p_0)$ was bounded away from $0$ must be wrong, and thus proves the result in Case I.

\subsection*{Case II}
We're in this case if the following inequality holds.
\begin{equation}
  \label{eq:up-8}
  0 < i(p_1, \lambda) < 1
\end{equation}
The picture in this case looks similar to \autoref{fig:hexagon}.
However, we can't necessarily pass to a subsequence where $f \geq \frac{1}{2}$ (and the trick of working with $1-f$ won't work, since we know nothing about $p_2$).
This is one of the points where the hypothesis on $p_1$ comes in.
Since $\frac{i(p_2, \lambda)}{i(p_1, \lambda)}$ is finite, we must have that the ratio of lengths $\frac{\ell_i(p_2)}{\ell_i(p_1)}$ approaches some finite value as well.
The fraction $f$ is a continuous function of $\frac{\ell_i(p_2)}{\ell_i(p_1)}$, approaching $0$ only as the ratio approaches $\infty$ (this follows from the same identity as \eqref{eq:pentagon}). Since the ratio approaches a finite value, we have a positive lower bound $f_0$ for $f$.

Assuming as before that $\ell_i(p_0)$ is bounded away from $0$, and $\tau(p_0)$ bounded away from $\pm \infty$, and repeating the calculations of the previous case, we get the following two inequalities.
\begin{equation}
  \label{eq:up-9}
  \frac{\ell_i(p_1)}{\ell_i(p_3)} \geq \frac{\ell_i(p_1) }{\ell_i(p_0) +
  2 \sinh^{-1} \left( \frac{\cosh\left( \frac{\ell_i(p_1)}{2} \right)}{f_0 \varepsilon} \right)
  }
\end{equation}
\begin{equation}
  \label{eq:up-10}
  \frac{\ell_i(p_0)}{\ell_i(p_3)} \geq \frac{\ell_i(p_0)}{\ell_i(p_0) +
  2 \sinh^{-1} \left( \frac{\cosh\left( \frac{\ell_i(p_1)}{2} \right)}{f_0 \varepsilon} \right)
  }
\end{equation}
The right hand side of \eqref{eq:up-10} must approach $0$, and that forces either $\ell_i(p_1)$ or $\ell_i(p_0)$ to approach $\infty$.
But that means the right hand term of \eqref{eq:up-9} must approach $1$, which cannot happen, by the hypothesis of case II.
This means $\ell_i(p_0)$ goes to $0$, proving the result in case II.

\subsection*{Reducing to case II} Suppose now that both $p_1$ and $p_2$ have an intersection number larger than $1$ with $\lambda$.
We can modify one of them to have a small intersection number with $\lambda$.
First, we assume that $\lambda_{\mathrm{Leb}}$ is supported on a single minimal component, i.e. every leaf of $\lambda_{\mathrm{Leb}}$ is dense in the support.
We now perform a local surgery on $p_1$: starting at a point on $p_1$ not contained in the support of $\lambda_{\mathrm{Leb}}$, we follow along until we intersect $\lambda_{\mathrm{Leb}}$ for the first time.
We denote this point by $\alpha$.
We now go along $p_1$ in the opposite direction, until we hit the support of $\lambda_{\mathrm{Leb}}$ again, but rather than stopping, we keep going until the arc has intersection number $0 < \delta < 1$ with $\lambda_{\mathrm{Leb}}$.
We then go back to $\alpha$, and follow along a leaf of $\lambda_{\mathrm{Leb}}$ rather than $p_1$, until we hit the arc.
This is guaranteed to happen by the minimality of $\lambda_{\mathrm{Leb}}$.
Once we hit the arc, we continue along the arc, and close up the curve.
This gives a new simple closed curve which intersection number with $\lambda$ is at most $\delta$.
This curve is our replacement for $p_1$.
If $\lambda_{\mathrm{Leb}}$ is not minimal, we repeat this process for each minimal component. We pick $p_2$ in a manner such that $p_0$, $p_1$, and $p_2$ bound a pair of pants.
Since $\delta < 1$, we have reduced to case II.
This concludes the proof of the theorem.
\end{proof}

\begin{remark}[On the orientable version of \autoref{prop:pinching}]
  The same idea also works in the orientable setting, although the analysis of the various cases gets a little more delicate.
  The first change one needs to make is in the statement of the proposition: we no longer need to require $p$ to be a one-sided atom, and correspondingly, either the length coordinate $\ell_i(p)$ can go to $0$, or the twist coordinate $\tau(p_0)$ can go to $\pm \infty$.
  To see how the twist coordinate enters the picture, observe that \eqref{eq:up-1}, which was the main inequality of the proof, turns into the following in the orientable version.
  \begin{equation}
    \label{eq:up-11}
    \ell_i(p_4)  \leq \ell_i(p_3) \leq \tau(p_0) + \ell_i(p_4)
  \end{equation}
  Here, $\tau(p_0)$ is the twist parameter about $p_0$, and $p_4$ is the orthogeodesic multi-arc (there may be one or two orthogeodesics, depending on the two cases described below).

  The proof splits up into two cases, depending on whether both sides of $p$ are the same pair of pants, or distinct pairs of pants.
  This was not an issue in the non-orientable setting, since $p$ was one-sided.
  If both sides of $p$ are the same pair of pants, then the analysis is similar to what we just did, since the curve $p_3$ stays within a single pair of pants.
  In the other, $p_3$ goes through two pair of pants, and its length is a function of the twist parameters, as well the cuff lengths of four curves, rather than two curves, the four curves being the two remaining cuffs of each pair of pants.
  The analysis again splits up into two cases, depending on the intersection number of the cuffs with $\lambda$, but reducing all the other cases to case II becomes tricky because we need to simultaneously reduce the intersection number of two curves, rather than one, as in the non-orientable setting.
  This added complication obscures the main idea of the proof, which is why we chose to only prove the non-orientable version.
\end{remark}

This quantitative estimate of \autoref{prop:pinching} gives us a proof for \autoref{thm:systole-closure}.
\begin{proof}[Proof of \autoref{thm:systole-closure}]
  Suppose that the theorem were false, and there was a foliation $[\lambda] \in \pmf^-(\no_g)$ in the closure of $\systole(\no_g)$.
  Suppose $p$ is a one-sided atom in $\lambda$.
  Then \autoref{prop:pinching} tells us that the hyperbolic length of $p$ goes to $0$, but the length of $p$ must be greater than $\varepsilon$ in $\systole(\no_g)$.
  This contradicts our initial assumption, and the closure of $\systole(\no_g)$ can only intersect $\pmf(\no_g)$ in the complement of $\pmf^-(\no_g)$.
\end{proof}

\begin{corollary}
  \label{cor:geolimset}
  The geometric limit set $\geolim(\mcg(\no_g))$ is contained in $\pmf^+(\no_g)$.
\end{corollary}
\begin{proof}
  Every point $p \in \teich(\no_g)$ is contained in $\systole(\no_g)$ for some small enough $\varepsilon$.
  This means $\Lambda_{\mathrm{geo}, p}(\mcg(\no_g))$ is contained in $\pmf^+(\no_g)$ by \autoref{thm:systole-closure}.
  Taking the union over all $p$ proves the result.
\end{proof}



%%% Local Variables:
%%% TeX-master: "main"
%%% End: