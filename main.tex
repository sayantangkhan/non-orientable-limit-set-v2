\documentclass[12pt, reqno]{amsart}
\pdfoutput=1
\usepackage{mathpreamble}

\title{The limit set of non-orientable mapping class groups}
\author{Sayantan Khan}
\address{Department of Mathematics, University of Michigan, Ann Arbor, MI}
\email{\href{mailto:saykhan@umich.edu}{saykhan@umich.edu}}
\thanks{}
\urladdr{\url{http://www-personal.umich.edu/~saykhan/}}

\keywords{mapping class group, Teichm\"uller space, Thurston boundary}
\subjclass[2010]{57K20}

\date{\today}

\begin{document}
\maketitle
\begin{abstract}

\end{abstract}
\tableofcontents

\section{Introduction}
\label{sec:introduction}

\section{Background on limit sets of mapping class groups}
\label{sec:backgr-limit-sets}

The first results on limit sets of subgroups of mapping class groups were obtained by Masur for handlebody subgroups \cite{masur_1986}, and McCarthy-Papadopoulos for general mapping class subgroups \cite{McCarthy1989}.
They defined two distinct notions of limit sets; while they did not give distinct names to the two different definitions, we will do so, for the sake of clarity.
\begin{definition}[Dynamical limit set]
  Given a subgroup $\Gamma$ of the mapping class group, the dynamical limit set $\dynlim(\Gamma)$ is the minimal closed invariant subset of $\pmf$ under the action of $\Gamma$.
\end{definition}
\begin{definition}[Geometric limit set]
  Given a subgroup $\Gamma$ of the mapping class group, and a point $x$ in the Teichmüller space, its boundary orbit closure $\Lambda_{\mathrm{geo}, x}(\Gamma)$ is intersection of its orbit closure with the Thurston boundary, i.e. $\overline{\Gamma x} \cap \pmf$.
  The geometric limit set is the union of all boundary orbit closures, as we vary $x$ in the Teichmüller space, i.e. $\geolim(\Gamma) = \bigcup_{x \in \teich} \Lambda_{\mathrm{geo}, x}(\Gamma)$.
\end{definition}

\begin{remark}
  The specific family of subgroups considered by McCarthy-Papadopoulos were subgroups containing at least two non-commuting pseudo-Anosov mapping classes, in which case, the dynamical limit set is unique.
  Unless otherwise specified, we will only talk about mapping class subgroups which contain at least two non-commuting pseudo-Anosovs.
\end{remark}

Both of these definitions are natural generalizations of the limit sets of Fuchsian groups acting on $\HH^2$.
In the hyperbolic setting, the two notions coincide, but for mapping class subgroups, the dynamical limit set is a proper subset of the geometric limit set.

For simple enough subgroups, one can explicitly work out $\dynlim(\Gamma)$ and $\geolim(\Gamma)$: for instance, when $\Gamma$ is the stabilizer of the Teichmüller disc associated to a Veech surface, $\dynlim(\Gamma)$ is the visual boundary of the Teichmüller, which by Veech dichotomy, only consists of either uniquely ergodic directions on the Veech surface, or the cylinder directions, where the coefficients on the cylinders are their moduli in the surface.
On the other hand, $\geolim(\Gamma)$ consists of all the points in $\dynlim(\Gamma)$, but it additionally contains all possible convex combinations of the cylinders appearing in $\dynlim(\Gamma)$ (see Section 2.1 of \cite{2007math......2034K}).

The gap between $\geolim$ and $\dynlim$ suggests the following expansion operation on subsets of $\pmf$
\begin{definition}[Expansion]
  Given a projective measured foliation $\lambda$, we define its expansion $\expansion(\lambda)$ to be the image in $\pmf$ of set of all non-zero measures invariant measures on the topological foliation associated to $\lambda$.
  Given a subset $\Lambda$, we define its expansion $\expansion(\Lambda)$ to be the union of expansions of the projective measured laminations contained in $\Lambda$.
\end{definition}
Observe that for a uniquely ergodic foliation $\lambda$, $\expansion(\lambda) = \{\lambda\}$, for a minimal but not uniquely ergodic $\lambda$, $\expansion(\lambda)$ is the convex hull of all the ergodic measures supported on the topological lamination associated to $\lambda$, and for a foliation with all periodic leaves, $\expansion(\lambda)$ consists of all foliations that can be obtained by assigning various weights to the core curves of the cylinders.

Going back to the example of the stabilizer of the Teichmüller disc of a Veech surface, we see that $\geolim(\Gamma) = \expansion(\dynlim(\Gamma))$.
One may ask if this is always the case.
\begin{question}
  Is $\geolim(\Gamma) = \expansion(\dynlim(\Gamma))$ for all $\Gamma$?
\end{question}

McCarthy-Papadopoulos also formulated an equivalent definition of $\dynlim(\Gamma)$, which is easier to work with in practice.

\begin{theorem}[Theorem 4.1 of \cite{McCarthy1989}]
  $\dynlim(\Gamma)$ is the closure in $\pmf$ of the stable and unstable foliations of all the pseudo-Anosov mapping classes in $\Gamma$.
\end{theorem}
A corollary of this theorem is that $\Gamma$ does not act properly discontinuously on $\dynlim(\Gamma)$.
In fact, one can construct a domain of discontinuity for the $\Gamma$ action using $\dynlim(\Gamma)$.
\begin{definition}[Zero intersection set]
  Given a subset $\Lambda$ of $\pmf$, its zero intersection set $Z(\Lambda)$ is the following subset of $\pmf$
  \begin{align*}
    Z(\Lambda) \coloneqq \left\{ \lambda^{\prime} \mid i(\lambda, \lambda^{\prime}) = 0 \text{ for some $\lambda \in \Lambda$} \right\}
  \end{align*}
\end{definition}

McCarthy-Papadopoulos also exhibit a domain of discontinuity for the $\Gamma$ action on $\pmf$.
\begin{theorem}[Theorem 6.16 of \cite{McCarthy1989}]
  The action of $\Gamma$ on $\pmf \setminus Z(\dynlim(\Gamma))$ is properly discontinuous.
\end{theorem}

It is not obvious that the action of $\Gamma$ on $Z(\dynlim(\Gamma)) \setminus \dynlim(\Gamma)$ is not properly discontinuous.
We prove that is indeed the case when $\Gamma = \mcg(\no_g)$ in \autoref{thm:dod-is-empty}, but the general case is unknown.

\section{$\mcg(\no_g)$ lower bound}
\label{sec:mcgno_g-lower-bound}

A natural lower bound for $\dynlim(\no_g)$ is the closure of the set of rational two-sided multicurves $\pmf^+(\no_g, \QQ)$.
For any $\lambda \in \pmf^+(\no_g, \QQ)$, and any psuedo-Anosov $\gamma$, conjugating $\gamma$ with large enough powers of the Dehn multi-twist given by $\lambda$ gives us a sequence of pseudo-Anosov maps whose stable foliation approaches $\lambda$, which shows that $\dynlim(\no_g)$ must contain $\lambda$.
Note that the same argument does not work if $\lambda \in \pmf^-(\no_g, \QQ)$, since one cannot Dehn twist about one-sided curves.
In \autoref{thm:systole-closure}, we show that the geometric limit set is indeed contained in the complement of $\pmf^-(\no_g)$.

In \cite{gendulphe_whats_2017}, Gendulphe made the following conjecture about $\overline{\pmf^+(\no_g, \QQ)}$.
\begin{conjecture}[Conjecture 9.1 of \cite{gendulphe_whats_2017}]
  For $g \geq 4$, $\pmf^+(\no_g) = \overline{\pmf^+(\no_g, \QQ)}$.
\end{conjecture}
We prove a slightly weaker version of the above conjecture, by describing which foliations can be approximated by foliations in $\pmf^+(\no_g, \QQ)$.
\begin{theorem}
  \label{thm:rational-approximation}
  For $g \geq 3$, a foliation $\lambda \in \pmf^+(\no_g)$ can be approximated by foliations in $\pmf^+(\no_g, \QQ)$ if all the minimal components $\lambda_i$ of $\lambda$ satisfy at least one of the following criteria.
  \begin{enumerate}[(i)]
  \item $\lambda_i$ is periodic.
  \item $\lambda_i$ is orientable.
  \item $\lambda_i$ is uniquely ergodic.
  \end{enumerate}
  Furthermore, if $\lambda_i$ is minimal, but not uniquely ergodic, there exists some other foliation $\lambda_i^{\prime}$ supported on the same topological foliation as $\lambda_i$ that can be approximated by elements of $\pmf^+(\no_g, \QQ)$.
\end{theorem}

To prove \autoref{thm:rational-approximation}, we will need to use the following lemmas.

\begin{lemma}
  \label{lem:valid-surgery}
  Let $\gamma_1$ and $\gamma_2$ be two-sided multicurves on $\no_g$ with $i(\gamma_1, \gamma_2) > 0$.
  For each intersection point of $\gamma_1$ and $\gamma_2$, there exists a choice of surgery to resolve the intersection such that the multicurve obtained by resolving all of the singularities contains only two-sided components.
\end{lemma}
\begin{proof}
  TODO
\end{proof}

\begin{lemma}
  \label{lem:convex-combinations}
  Let $\gamma_1$ and $\gamma_2$ be two measured foliations supported on the same topological foliation such that the transverse measures induced by $\gamma_1$ and $\gamma_2$ are mutually singular.
  Furthermore, suppose both $\gamma_1$ and $\gamma_2$ are approximable by simple two-sided multicurves.
  Then any convex combination $c_1 \gamma_1 + c_2\gamma_2$ is also approximable by simple two-sided multicurves.
\end{lemma}

\begin{proof}
  TODO
\end{proof}

% \paragraph{Idea of proof}
% First of all, note that one can approximate each minimal component on the subsurface on which the minimal components live in, and combine the rational approximations.
% This works because the rational approximations are also on difference subsurfaces, and thus have zero intersection number with each other.
% Case (i) is easy to deal with: since $\lambda_i \in \pmf^+(\no_g)$, the core curve is two-sided.
% Approximating that two-sided curve with the same two-sided curve, but with rational weights does the job.
% Cases (ii) and (iii) require more work.
% In case (ii), we first approximate the ergodic measures supported on the topological foliation $\lambda_i$.
% We do so by following a long orbit that equidistributes with respect to the ergodic measure, and closing the orbit up so that the resulting curve is simple and two-sided.
% To close up the orbit so that it still equidistributes with the same ergodic measure, we need to ensure that the foliation is orientable.
% Once we have rational approximations for all the ergodic measures supported on the topological foliation $\lambda_i$, we can obtain rational approximations for any convex combinations by taking large enough Dehn twists about each of the rational approximants.
% That proves the result for case (ii).
% For case (iii), we do the same thing as in case (ii), i.e. close up a long orbit so that it is simple and two-sided.
% Unique ergodicity ensures that the resulting curve equidistributes with respect to the uniquely ergodic measure.

\section{$\mcg(\no_g)$ upper bound}
\label{sec:mcgno_g-upper-bound}

\section{Domain of discontinuity}
\label{sec:domain-discontinuity}

\section{Failure of quasi-convexity for $\systole$}
\label{sec:fail-quasi-conv}



\printbibliography

\end{document}
