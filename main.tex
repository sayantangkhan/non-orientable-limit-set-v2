\documentclass[12pt, reqno]{amsart}
\pdfoutput=1
\usepackage{mathpreamble}

\title{The limit set of non-orientable mapping class groups}
\author{Sayantan Khan}
\address{Department of Mathematics, University of Michigan, Ann Arbor, MI}
\email{\href{mailto:saykhan@umich.edu}{saykhan@umich.edu}}
\thanks{}
\urladdr{\url{http://www-personal.umich.edu/~saykhan/}}

\keywords{mapping class group, Teichm\"uller space, Thurston boundary}
\subjclass[2010]{57K20}

\date{\today}

\begin{document}
\maketitle
\begin{abstract}
  Motivated by the similarities between geometrically finite infinite covolume Fuchsian groups and the mapping class group of compact non-orientable surfaces, we show that the limit set of the mapping class group is contains the set of uniquely ergodic foliations and is contained in the set of all projective measured foliations not containing any one-sided leaves.
  Furthermore, we show that the image of limit set in the space of laminations contains all ending laminations.
  This provides strong evidence for a conjecture of Gendulphe.
  Using the limit set we also show that a conjectured convex core for the moduli space is in fact not even quasi-convex.
\end{abstract}
\tableofcontents

\section{Introduction}
\label{sec:introduction}

\section{Closure of $\mf^+(\no_d, \Q)$}
\label{sec:closure-mf}

In this section, we show that the closure of the set of rational weighted two-sided curves $\mf^+(\no_d, \Q)$ in the set of all measured foliations $\mf(\no_d)$ contains all the measured foliations whose minimal components are uniquely ergodic, or rational two-sided multicurve.
The idea of the proof is the following: given a uniquely ergodic measured foliation, we consider the first return map to a transverse segment.
The first return map is a \emph{linear involution with flips} (abbreviated LIF); we construct a rational approximation to the LIF such that all the intervals are periodic with even period.
The even parity of the periods ensures that the corresponding weighted multicurve approximation to the given foliation only has two-sided curves, which then proves the claim.

\subsection*{Linear involutions with flips and Rauzy induction}
% \label{sec:foli-induc-fiets}

A linear involution with flips is a dynamical system on the union of two intervals of equal length given by the following data.
\begin{description}
\item[Alphabet $\cA \cup \cA^{\prime}$]: This the disjoint union of two copies of the same finite set $\cA$ of cardinality $n$: for any $\alpha \in \cA$, its copy in $\cA^{\prime}$ is denoted $\alpha^{\prime}$.
\item[Partition $(p,q)$]: This is an ordered pair of positive integers such that $p+q = 2n$.
\item[Ordering $\pi$]: This is a bijective map from $\{1, \ldots, 2n\}$ to $\cA \cup \cA^{\prime}$.
\item[Flip set $F$]: This is a subset of $\cA \cup \cA^{\prime}$ such that $\alpha \in F$ iff $\alpha^{\prime} \in F$.
\item[Length vector $\ell$]: This a vector in $\mathbb{R}^{\cA \cup \cA^{\prime}}$ all of whose coordinates are positive.
  The coordinates of the length vector also must satisfy two constraints.
  The first constraint is that lengths of copies of intervals must be equal, i.e. for all $\alpha$ in $\cA$, the following holds.
  \begin{align*}
    \ell_{\alpha} &= \ell_{\alpha^{\prime}}
  \end{align*}
  The second constraint dictates that the total length of the two partitions must be equal.
  \begin{align*}
    \sum_{i=1}^{p} \ell_{\pi(i)} = \sum_{i=p+1}^{p+q} \ell_{\pi(i)}
  \end{align*}
\end{description}
The space on which this dynamical system acts is obtained by first taking the disjoint unions of intervals indexed by $\cA \cup \cA^{\prime}$ of length $\ell_y$ as $y$ varies over $\cA \cup \cA^{\prime}$.
These intervals are concatenated in the order $\{\pi(1), \ldots, \pi(p)\}$ to form the \emph{left interval}, and then in the order $\{\pi(p+1), \ldots, \pi(p+q)\}$ to form the right interval.
The disjoint union of the left and right intervals is the space on which the linear involution acts.
One iteration of the linear involution is given by swapping every point in the interior of the interval corresponding to $\alpha$ and $\alpha^{\prime}$, for all $\alpha \in \cA$, and then swapping the left and right interval.

This reduction of the flow of a measured foliation to an LIF is well understood (see \cite{ASENS_1990_4_23_3_469_0} and \cite{boissy2009dynamics} for details), and provides a systematic way of understanding the dynamics of the flow associated to the foliation.
The following two lemmas encapsulate the reduction of the problem of approximating minimal foliations to a problem about linear involutions with flips.
\begin{lemma}
  If $(\cA \cup \cA^{\prime}, (p,q), \pi, F, \ell)$ is an LIF obtained as the first return map of a measured foliation, then $( \cA \cup \cA^{\prime}, (p,q), \pi, F, \ell^{\prime})$ also corresponds to a measured foliation.
  Furthermore if $\ell^{\prime}$ converges to $\ell$, then the associated foliations converge to the original foliation as well.
\end{lemma}
\begin{proof}
  The proof follows easily from Veech's zippered rectangle construction (see section 1.3.5 of \cite{boissy2009dynamics}).
\end{proof}

\begin{lemma}
  The foliation associated to $( \cA \cup \cA^{\prime}, (p,q), \pi, F, \ell^{\prime})$ is a rational two sided multicurve iff $\ell^{\prime}$ has all rational coordinates, and every periodic point of the LIF returns after passing through an even number of intervals that are in the flip set $F$.
\end{lemma}
\begin{proof}
  The vertical trajectories along the periodic points of the LIF correspond to the curves in the rational multicurve.
  If some periodic point passes through an odd number of flipped intervals, then a tubular neighbourhood of that must be a M\"obius strip, and thus would correspond to a one-sided curve in the multicurve.
  Conversely, if a periodic point passes through an even number of intervals, its tubular neighbourhood must be a cylinder, and thus a two-sided curve.
\end{proof}

For the rest of this section, we will focus on uniquely ergodic LIFs, which correspond exactly to the uniquely ergodic foliations.
Given a uniquely ergodic LIF, we would like to understand the first return map to a smaller pair of sub-intervals, and understand the dynamics of the points of the smaller interval as they go through the intervals of the original LIF.
Our main tools for doing this will be \emph{Rauzy induction} and the associated \emph{Rauzy cocycle}.
\begin{definition}[Rauzy induction]
  Given an LIF $T = (\cA \cup \cA^{\prime}, (p,q), \pi, F, \ell)$, Rauzy induction is a map $\cR$ that sends $T$ to $\cR(T)$, where $\cR(T)$ is the linear involution obtained by considering the first return map on the pair of subintervals obtained by taking the subintervals $(0, I - \min(\ell_{\pi(p)}, \ell_{\pi(p+q)}))$ of both the left and the right intervals of the original LIF. Here $I$ is the length of the left and right intervals of the original LIF.
\end{definition}
\begin{remark}
  The definitions of the map $\cR$ in \cite{ASENS_1990_4_23_3_469_0} and \cite{boissy2009dynamics} leave it undefined when $\ell_{\pi(p)} = \ell_{\pi(p+q)}$, since the first return map to the subinterval in this case is an LIF on an alphabet of size $2n-2$ rather than $2n$ as in the other cases.
  However, for our purposes, the drop in the alphabet size is not a problem, so we define $\cR$ for all the cases.
\end{remark}

\section{Closure of $\systole$}
\label{sec:closure-systole}

\section{Failure of quasi-convexity for $\systole$}
\label{sec:fail-quasi-conv}



\printbibliography

\end{document}
