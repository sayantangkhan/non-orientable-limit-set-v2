\documentclass[12pt, reqno]{amsart}
\pdfoutput=1
\usepackage{mathpreamble}

\title{The limit set of non-orientable mapping class groups}
\author{Sayantan Khan}
\address{Department of Mathematics, University of Michigan, Ann Arbor, MI}
\email{\href{mailto:saykhan@umich.edu}{saykhan@umich.edu}}
\thanks{}
\urladdr{\url{http://www-personal.umich.edu/~saykhan/}}

\keywords{mapping class group, Teichm\"uller space, Thurston boundary}
\subjclass[2010]{57K20}

\date{\today}

\begin{document}

\begin{abstract}
  Motivated by the conjectured similarities between geometrically finite infinite covolume Fuchsian groups and the mapping class group of compact non-orientable surfaces, we show that the limit set of the mapping class group contains the set of uniquely ergodic foliations and is contained in the set of all projective measured foliations not containing any one-sided leaves.
  This provides strong evidence for a conjecture of Gendulphe.
  Using the limit set we also show that a conjectured convex core for the moduli
  space is in fact not even quasi-convex, and the action of the mapping class group on the boundary has an empty domain of discontinuity despite the complement of the limit set being an open dense subset.
\end{abstract}
\maketitle

% \tableofcontents

\section{Introduction}
\label{sec:introduction}


The moduli space $\mg(\no_g)$ of compact \emph{non-orientable} hyperbolic surfaces of genus $g$ is conjectured to have similarities to infinite volume geometrically finite manifolds (in a manner similar to how moduli spaces of compact orientable surfaces have properties similar to finite volume hyperbolic manifolds).
The main results suggesting the analogy between moduli spaces of non-orientable surfaces and infinite volume manifolds are due to Norbury and Gendulphe.

\begin{itemize}
\item The $\mg(\no_g)$ has infinite Teichm\"uller volume (Theorem 17.1 of \cite{gendulphe_whats_2017}).
  While the associated Teichm\"uller space does not have a Weil-Peterson volume form, it has an analogous volume form with respect to which the moduli space has infinite volume as well (see \cite{norbury2008lengths}).
\item The action of the mapping class group $\mcg(\no_g)$ on the Thurston boundary is not minimal (Proposition 8.9 in \cite{gendulphe_whats_2017}).
\item The Teichm\"uller geodesic flow is not topologically transitive, and thus not ergodic with respect to any Borel measure with full support (Proposition 17.5 of \cite{gendulphe_whats_2017}).
\end{itemize}

We attempt to gain a better understanding of this conjectured analogy by computing the limit set in $\pmf(\no_g)$ of $\mcg(\no_g)$.
We show that the limit set contains a large class of projective measured foliations that do not have a one-sided leaf.

\begingroup
\def\thetheorem{\ref{thm:rational-approximation}}
\begin{theorem}
  For $g \geq 3$, a foliation $\lambda \in \pmf^+(\no_g)$ is in the limit set of $\mcg(\no_g)$ if all the minimal components $\lambda_j$ of $\lambda$ satisfy one of the following criteria.
  \begin{enumerate}[(i)]
  \item $\lambda_j$ is periodic.
  \item $\lambda_j$ is orientable, i.e. all leaves exiting one side of a transverse arc always come back from the other side.
  \item $\lambda_j$ is uniquely ergodic.
  \end{enumerate}
  Furthermore, if $\lambda_j$ is minimal, but not uniquely ergodic, there exists some other foliation $\lambda_j^{\prime}$ supported on the same topological foliation as $\lambda_j$ which is in the limit set.
\end{theorem}
\addtocounter{theorem}{-1}
\endgroup

We also show that the limit set is contained in the complement of the open dense set of foliations that contain a one-sided leaf.
\begingroup
\def\thetheorem{\ref{cor:geolimset}}
\begin{theorem}
  The limit set is contained in the complement of the set of foliations that contain a one-sided leaf.
\end{theorem}
\addtocounter{theorem}{-1}
\endgroup
These two results together give an almost complete proof of Conjecture 9.1 in \cite{gendulphe_whats_2017}.

We outline the key ideas used in the proof of the above two theorems.
The key idea in the proof of \autoref{thm:rational-approximation} is picking a short transverse arc to the foliation, and studying the return map to the arc.
By shrinking the arc on successive returns, one can close up the path to a \emph{simple} closed curve, and then use ergodicity to claim that the simple closed curve indeed approximates the foliation.
The key step in the proof of Theorem \ref{cor:geolimset} is obtaining precise estimates on the Fenchel-Nielsen coordinates of points diverging to the boundary of the Teichm\"uller space, and showing the length coordinates of one-sided curves go to $0$.

Gendulphe also conjectured that an analogy might be made with geometrically finite hyperbolic manifolds.
They constructed a finite volume $\mcg(\no_g)$ equivariant retraction of the Teichm\"uller space to $\systole$, the set of points in the Teichm\"uller space that have no one-sided curves shorter than $\varepsilon$.
They then asked whether $\systole$ was a quasi-convex subset of $\teich(\no_g)$: if it were, its image in the moduli space would be a finite volume convex core, just like convex cores of geometrically finite manifolds.
We show that $\systole$ is actually not quasi-convex in the strong sense, answering Question 19.1 of \cite{gendulphe_whats_2017}.
\begingroup
\def\thetheorem{\ref{thm:qc-fail}}
\begin{theorem}
  For all $\varepsilon > 0$, and all $D > 0$, there exists a Teichm\"uller geodesic whose endpoints lie in $\overline{\systole} \cap \pmf$ such that some point in interior of the geodesic is more than distance $D$ from $\systole$.
\end{theorem}
\addtocounter{theorem}{-1}
\endgroup
The negative result suggests that in order to make the analogy work in a robust manner, we may need to relax some of the requirements we require from the convex hull of the limit set. In particular, we may need to discard some of the limit points, the ones analogous to parabolic limit points, and focus on the limit points that are analogous to radial limit points in the hyperbolic setting.
We venture this guess since the limit points involved in proof of \autoref{thm:qc-fail} are stabilized by Dehn multi-twists, which are analogous to parabolic elements of Fuchsian groups.

We also prove another result that indicates that the analogy with infinite volume geometrically finite manifolds might not be as direct as we would like.
In the case of infinite volume geometrically finite manifolds, the domain of discontinuity in the boundary is a large open set contained in the complement of the limit set.
In the setting of $\teich(\no_g)$, even though the limit set is contained in the complement of a large open set, we show that the domain of discontinuity is empty.
\begingroup
\def\thetheorem{\ref{thm:dod-is-empty}}
\begin{theorem}
  The domain of discontinuity for the $\mcg(\no_g)$ action on $\pmf(\no_g)$ is empty.
\end{theorem}
\addtocounter{theorem}{-1}
\endgroup
This result is in stark contrast with the result for convex-cocompact subgroups of mapping class groups of orientable, where it is known that the action on the complement of the limit set is properly discontinuous (see \cite{farb2002convex}).

Apart from the conjectured analogy being inherently interesting, these questions also show up when counting simple closed geodesics of length less than $L$ on any hyperbolic surface. Mirzakhani, in \autocite{mirzakhani2008growth}, obtained precise asymptotics for this counting function in the orientable case by essentially proving equidistribution (with respect to Thurston measure) of $\mcg(\os_{g})$-orbits in $\pmf(\os_g)$.
To do the same for the non-orientable case, we need an ergodic measure supported on sets
minimal with respect to the $\mcg$ action, e.g. $\overline{\pmf^+(\no_d)}$.
One way to construct such a measure would be to replicate the construction of Patterson-Sullivan measures for geometrically finite manifolds, which brings us back
to the analogy between $\mg(\no_d)$ and infinite volume geometrically finite manifolds.

\subsection*{Context}

The author is aware of two other subfields of Teichm\"uller theory that naturally lead to the study of the Teichm\"uller space of non-orientable surfaces.

The first concerns isometrically embedded submanifolds of the Teichm\"uller space of orientable surfaces.
The simplest of such manifolds is just a Teichm\"uller geodesic, and these are fairly well
understood objects.
The simplest isometrically embedded complex submanifolds are \emph{Teichm\"uller
  discs}, which are isometric to $\mathbb{H}^2$.
On these isometrically embedded subsets, the analogy with hyperbolic space is perfect, since these are isometric to $\mathbb{H}^2$, and as a consequence, the dynamics on the quotient can be understood using tools from hyperbolic geometry.
The stabilizers of these can vary in size, from being as large a lattices, to as small as infinitely generated free groups, and conjecturally, as small as a span of a single pseudo-Anosov element.
The higher dimensional complex submanifolds exhibit more rigidity (see \cite{wright2020}), and it is expected that the limit set of their stabilizer is a sphere of real dimension $2d-1$ (where $d$ is the complex dimension of the submanifolds), i.e. as big as it can get. Examples of higher dimensional totally real isometrically embedded submanifolds come from Teichm\"uller spaces of non-orientable surfaces.
The limit set of the stabilizers of these manifolds is smaller in comparison to the complex variants.

The second way Teichm\"uller spaces of non-orientable surfaces crop up is through \emph{interval exchange transformations with flips}.
The dynamics of interval exchange transformations are closely related to the dynamics of horizontal/vertical flow on an associated quadratic differential, which is related to the geodesic flow on the Teichm\"uller surface via Masur's criterion (a version of which holds in the non-orientable setting as well).
IETs with flips do not have very good recurrence properties: in fact, almost all of them (with respect to the Lebesgue measure) have a periodic point (see \cite{nogueira_1989}) and the set of minimal IETs with flips have a lower Hausdorff dimension (see \cite{skripchenko2018hausdorff}).
To understand the IETs which are uniquely ergodic, one is naturally led to determine which ``quadratic differentials'' on non-orientable surfaces are recurrent.
A necessary but not sufficient condition for recurrence of a Teichm\"uller geodesic is that its forward and backward limit points lie in the limit set.
From this perspective, Theorems \ref{thm:rational-approximation} and \ref{cor:geolimset} can
be seen as a statement about the closure of the recurrent set.
Constructing a measure supported on the closure of the recurrent set can be then used to answer questions about uniquely ergodic IETs with flips.

\subsection*{Organization of the paper}
Section \ref{sec:backgr-meas-foli} contains the background on non-orientable surfaces and measured foliations, and section \ref{sec:backgr-limit-sets} contains the background on limit sets of mapping class subgroups.
These sections can be skipped and later referred to if some notation or definition is unclear.
Section \ref{sec:lower-bound-limit-set} contains the proof of \autoref{thm:rational-approximation}, section \ref{sec:upper-bound-limit-set} contains the proof of Theorem \ref{cor:geolimset}, section \ref{sec:domain-discontinuity} contains the proof of \autoref{thm:dod-is-empty}, and section \ref{sec:fail-quasi-conv} contains the proof of \autoref{thm:qc-fail}.
Sections \ref{sec:lower-bound-limit-set}, \ref{sec:upper-bound-limit-set}, and \ref{sec:fail-quasi-conv} are independent of each other, and can be read in any order, but section \ref{sec:domain-discontinuity} relies on the notation and results from section \ref{sec:lower-bound-limit-set}.

% \subsection*{Acknowledgments}
% {\color{red} To insert.}

\section{Background}
\label{sec:background}

\subsection{Non-orientable surfaces and measured foliations}
\label{sec:backgr-meas-foli}

For the purposes of this paper, the most convenient way to think about non-orientable surfaces will be to attach \emph{crosscaps} to orientable surfaces.
Given a surface $S$, attaching a crosscap is the operation of deleting the interior of a small embedded disc, and gluing the boundary $S^1$ via the antipodal map.
Attaching $k$ crosscaps to a genus $g$ surface results in a genus $2g+k$ non-orientable surface $\no_{2g+k}$ (i.e. the non-orientable surface obtained by taking the connect sum of $2g+k$ copies of $\mathbb{RP}^2$).
% In particular, any non-orientable surface can be obtained by just attaching $1$ or $2$ crosscaps to an orientable surface.

Consider the set $\scc$ of simple closed curves on a non-orientable surface $\no$.
The elements of $\scc$ can be classified into two types.
\begin{description}
\item[Two sided curves] Tubular neighbourhoods are cylinders.
\item[One sided curves] Tubular neighbourhoods are Möbius bands.
\end{description}
The subset of two sided curves in denoted by $\scc^+$ and one sided curves by $\scc^-$.
Since these two types are topologically distinct, they form invariant subspaces with respect to the mapping class group action.

The orientable double cover of $\no_g$ is the orientable surface $\os_{g-1}$, and comes with an orientation reversing involution $\iota$.
Since this is an orientation double cover, the subgroup of $\pi_1(\no_g)$ corresponding to this cover is characteristic, i.e. left invariant by every homeomorphism induced automorphism of the fundamental group.
A useful consequence of this fact is that one can lift mapping classes uniquely.
\begin{fact}
  Any self homeomorphism of $\no_g$ lifts to a unique orientation preserving self homeomorphism of $\os_{g-1}$, and as a consequence, one has the injective homomorphism induced by the covering map $p$.
  \begin{align*}
    p^{\ast}: \mcg(\no_d) \hookrightarrow \mcg^+(\os_{d-1})
  \end{align*}
  Furthermore, this inclusion preserves the mapping class type, i.e. finite order, reducible and pseudo-Anosov maps in $\mcg(\no_g)$ stay finite order, reducible, and pseudo-Anosov in $\mcg(\os_{g-1})$.
\end{fact}

One also obtains a map from $\teich(\no_g)$ to $\teich(\os_{g-1})$ using the fact that mapping classes can be lifted canonically.
Given a point $(p, \varphi)$ in $\teich(\no_g)$, where $p$ is a hyperbolic surface homeomorphic to $\no_g$, and $\varphi$ is an isotopy class of homeomorphism from $\no_g$ to $p$, we define the image of $(p, \varphi)$ in $\teich(\os_{g-1})$ to be $(\widetilde{p}, \widetilde{\varphi})$, where $\widetilde{p}$ is the orientation double cover of $p$, and $\widetilde{\varphi}$ is the orientation preserving lift of the homeomorphism $\varphi$.
One can also explicitly describe the image of this map.
To do so, we consider the extended Teichmüller space of $\os_{g-1}$, i.e. also allowing orientation reversing markings.
This space has two connected components, one for each orientation, and there is a canonical involution, given by reversing the orientation, that exchanges the two connected components.
We denote this conjugation map by $\overline{\cdot}$.
There is another involution, induced by the orientation reversing deck transformation of $\os_{g-1}$, which we denote by $\iota^{\ast}$.
This map also exchanges the two components of the extended Teichmüller space.
The image of $\teich(\no_g)$ is precisely the set of points fixed by the composition of these two maps, i.e. $\overline{\iota^{\ast}}$.
We skip the proof of these two facts, since they follow by relatively elementary covering space arguments, and summarize the result in the following theorem.

\begin{theorem}[Embedding Teichm\"uller spaces]
  \label{thm:embedding-teich}
  Given a point $(p, \varphi)$ in $\teich(\no_g)$, there is a unique point $(\widetilde{p}, \widetilde{\varphi})$ in $\teich(\os_{g-1})$, where $\widetilde{p}$ is the pullback of the metric, and $\widetilde{\varphi}$ is the unique orientation preserving lift of the marking. The image of the inclusion map is the intersection of the invariant set of $\overline{\iota^\ast}$ with the connected component of the extended Teichmüller space corresponding to orientation preserving maps.
\end{theorem}

It turns out that the image of $\teich(\no_g)$ in $\teich(\os_{g-1})$ is an isometrically embedded submanifold, and the geodesic flow can be represented by the action of the diagonal subgroup of $SL(2, \mathbb{R})$.

To understand the Teichm\"uller geodesic flow on $\teich(\no_g)$, we need to determine what the cotangent vectors look like: let $X$ be a point in $\teich(\no_{g})$ and let $\wt{X}$ be the corresponding point in $\teich(\os_{g-1})$.
Then the map on the extended Teichmüller space induced by the orientation reversing deck transformation maps $\wt{X}$ to $\overline{\wt{X}}$, i.e. the conjugate Riemann surface. Following that with the canonical conjugation map brings us back to $\wt{X}$.
Let $q$ be a cotangent vector at $\overline{\wt{X}}$, i.e. an anti-holomorphic quadratic differential on the Riemann surface $\overline{\wt{X}}$.
Pulling back $q$ along the canonical conjugation map gives a holomorphic quadratic differential on $X$.
In local coordinate chart on $\wt{X}$, this looks like $q(z) dz^2$ if on the corresponding chart on $\overline{\wt{X}}$ it looked like $q(\overline{z}) dz^2$.
We want this to equal $\iota^{\ast}q$, which will also be a holomorphic quadratic differential on $\wt{X}$.
If that happens, then $\iota^{\ast} q$ is a cotangent vector to the point $X$ in $\teich(\no_g)$.

\begin{example}[A cotangent vector to a point in $\teich(\no_3)$]
  Consider the quadratic differential $q$ on a genus two Riemann surface pictured in \autoref{fig:dqd-example}.
  \begin{figure}[h]
    \centering
    \incfig[0.5]{dqd-example}
    \caption{A quadratic differential $q$ on $\os_2$ given by the slit torus construction.}
    \label{fig:dqd-example}
  \end{figure}

  Observe that this particular quadratic differential is the global square of an abelian differential, so it makes sense to talk about the pairing between $\sqrt{q}$ and the homology classes $\{a, a^{\prime}, b, b^{\prime}, c, c^{\prime}\}$.
  Recall that the action of a mapping class like $\iota$ is merely relabelling homology classes: in this case $\iota$ swaps $a$ with $-a^{\prime}$, $b$ with $b^{\prime}$, and $c$ with $-c^{\prime}$.
  That gives us the following expressions involving $\sqrt{q}$.
  \begin{align}
    \label{eq:back-12}
    \langle \iota^{\ast}\sqrt{q}, a \rangle &= \langle \sqrt{q}, -a^{\prime} \rangle \\
    \langle \iota^{\ast}\sqrt{q}, b \rangle &= \langle \sqrt{q}, b^{\prime} \rangle \\
    \langle \iota^{\ast}\sqrt{q}, c \rangle &= \langle \sqrt{q}, -c^{\prime} \rangle
  \end{align}
  On the other hand, the conjugation action conjugates the complex value of each pairing.
  \begin{align}
    \label{eq:back-13}
    \langle \overline{\sqrt{q}}, a \rangle &= \overline{  \langle \sqrt{q}, a \rangle } \\
    \langle \overline{\sqrt{q}}, b \rangle &= \overline{  \langle \sqrt{q}, b \rangle } \\
    \langle \overline{\sqrt{q}}, c \rangle &= \overline{  \langle \sqrt{q}, c \rangle }
  \end{align}
  For $q$ to be invariant under $\overline{\iota}$, both of the above set of equations must be satisfied, which imposes certain conditions on $q$.
  For instance, if the complex lengths of $a$ and $a^{\prime}$ to be conjugates of each other, the complex lengths of $b$ and $b^{\prime}$ to be negative conjugates of each other, and forces the complex length of $c$ and $c'$ to be real.
  Only the quadratic differentials satisfying these constraints will be the cotangent vectors to points in the image of $\teich(\no_3)$.

  To realize the quadratic differential directly as an object on $\no_3$, we can quotient out the flat surface given by $q$ by the orientation reversing deck transformation.
  Doing that for our example gives the non-orientable flat surface gives the picture seen in \autoref{fig:dqd-no3}.
  \begin{figure}[h]
    \centering
    \incfig[0.5]{dqd-no3}
    \caption{A quadratic differential on $\no_3$.}
    \label{fig:dqd-no3}
  \end{figure}
\end{example}
This example suggests what the right definition of a quadratic differential on a non-orientable surface ought to be: in the flat picture, rather than allowing gluing via just the maps $z \mapsto \pm z + c$, we also allow $z \mapsto \pm \overline{z} + c$.
This leads to the definition of \emph{dianalytic quadratic differentials} (which we'll abbreviate to DQDs).

\begin{definition}[Dianalytic quadratic differential (adapted from \autocite{Wright2015})]
  A dianalytic quadratic differential is the quotient of a collection of polygons in $\mathbb{C}$, modulo
  certain equivalences.  The quotienting satisfies the following conditions.
  \begin{enumerate}[(1)]
  \item The interiors of the polygons are disjoint.
  \item Each edge is identified with exactly one other edge, and the mapping must be of one of the following
    four forms: $z \mapsto z + c$, $z \mapsto -z + c$, $z \mapsto \overline{z} + c$, or
    $z \mapsto -\overline{z} + c$.
  \item Extending the edge identification map to a small enough open neighbourhood of a point on the edge
    should not map it to an open neighbourhood of the image of the point: in other words, it should get mapped
    to the ``other side'' of the edge.
  \end{enumerate}
  Two such quotiented collections of polygons are considered the same if they differ by a composition of the following
  moves.
  \begin{enumerate}[(1)]
  \item A polygon may be translated, rotated by $\pi$ radians, or reflected across the real or imaginary axis.
  \item A polygon may be cut along a straight line to form two polygons, or two polygons sharing an edge may
    be glued together to form a single polygon.
  \end{enumerate}
\end{definition}

Given a DQD, we can pull it back to the orientation double cover, getting an actual quadratic differential: this operation corresponds to identifying a cotangent vector to a point in $\teich(\no_g)$ to the corresponding cotangent vector in $\teich(\os_{g-1})$.

To verify that $\teich(\no_g)$ is isometrically embedded, all we need to do is verify that the Teichm\"uller geodesic flow takes the quadratic differentials satisfying the symmetry condition $\iota^{\ast}(q) = \overline{q}$ to quadratic differentials that satisfy the symmetry conditions.

\begin{lemma}
  \label{lem:gt-invariance}
  If $q$ satisfies $\iota^{\ast}(q) = \overline{q}$, then for any $t$, $\iota^{\ast}(g_tq) = \overline{g_tq}$.
\end{lemma}
\begin{proof}
  Recall that if $q$ satisfies the given condition, we must have the following hold for any homology class $a$.
  \begin{equation}
    \label{eq:back-14}
    \langle \sqrt{q}, \iota(a) \rangle = \overline{ \langle \sqrt{q}, a \rangle}
  \end{equation}
  If $q$ is not the global square of an abelian differential, we may have to pass to the holonomy double
  cover. Observe now what $g_t$ does to $q$.
  \begin{equation}
    \label{eq:back-15}
    \langle \sqrt{g_t q}, \iota(a) \rangle = e^t \mathrm{Re} \langle \sqrt{q}, \iota(a) \rangle + i e^{-t} \mathrm{Im}\langle \sqrt{q}, \iota(a) \rangle
  \end{equation}
  Using \eqref{eq:back-14}, we simplify \eqref{eq:back-15} to the following.
  \begin{align}
    \label{eq:back-16}
    \langle \sqrt{g_t q}, \iota(a) \rangle &= e^t \mathrm{Re} \langle \sqrt{q}, a \rangle - i e^{-t} \mathrm{Im}\langle \sqrt{q}, a \rangle \\
                                           &= \overline{\langle \sqrt{g_tq}, a \rangle}
  \end{align}
  This proves the lemma.
\end{proof}
\begin{remark}
  The key idea that diagonal matrices commute:
  the conjugation action is really multiplication by $
  \begin{pmatrix}
    1 & 0 \\
    0 & -1
  \end{pmatrix}
  $ which happens to commute with the diagonal matrices of determinant $1$, which are exactly the matrices
  corresponding to geodesic flow. On the other hand, the conjugation matrix does not commute with the horocycle
  flow matrices, and that shows that the horocycle flow is not well defined on the cotangent bundle
  of $\teich(\no_g)$.
\end{remark}
Lemma \ref{lem:gt-invariance} shows that the Teichm\"uller geodesic flow for the cotangent bundle of $\teich(\no_g)$ is the restriction of the geodesic flow for the ambient space $\teich(\os_{g-1})$.

\autoref{thm:embedding-teich} gives us an alternative perspective into the action of $\mcg(\no_g)$ on $\teich(\no_g)$.
$\mcg(\no_g)$ can be thought of as the subgroup of $\mcg(\os_{g-1})$ that stabilizes a totally real isometrically embedded submanifold $\teich(\no_g)$.
With this perspective, $\mcg(\no_g)$ can be thought of as the higher dimensional generalization of the subgroups obtained by stabilizing Teichmüller discs, i.e. Veech groups.

We now state a few classical results about measured foliations on non-orientable surfaces that show why the theory diverges significantly from the orientable case.

A measured foliation on a non-orientable surface $\no_g$ is singular foliation along with an associated transverse measure, up to equivalence by Whitehead moves.
Any leaf of a measured foliation can either be non-compact or compact: in the former case, the closure of the non-compact leaf fills out a subsurface.
Restricted to the subsurface given by the closure of a non-compact leaf, the foliation is minimal, i.e. the orbit of every point under the flow given by the foliation is dense.
For a compact leaf, there are two possibilities for the topology of the subsurface containing it: if the closed leaf is the core curve or the boundary curve of an embedded M\"obius strip, then the subsurface is the maximal neighbourhood of the periodic leaf that is foliated by periodic leaves as well, and this turns out to be an embedded M\"obius strip.
If the compact leaf is not the core curve or the boundary curve of an embedded M\"obius strip, then it is the core curve of an embedded cylinder, and the maximal neighbourhood of the periodic leaf foliated by periodic leaf is an embedded cylinder.
The identification of leaves with associated subsurfaces lets us decompose a measured foliation into its minimal components.
Note the slightly confusing terminology: when the minimal component is a M\"obius strip or a cylinder, then the foliation restricted to the component is not minimal, but when the minimal component has higher genus, then the foliation restricted to that component indeed is minimal.

We denote the set of measured foliations on $\no_g$ by $\mf(\no_g)$, the set of foliations whose minimal components do not contain a M\"obius strip by $\mf^+(\no_g)$, and the set of foliations whose minimal components contain at least one M\"obius strip by $\mf^-(\no_g)$.
Via the standard identification between simple closed curves and measured foliations, we can associate $\mathbb{Q}$-weighted two-sided multicurves on $\no_g$ to a subset of $\mf^+(\no_g)$, denoted by $\mf^+(\no_g, \mathbb{Q})$.

Quotienting out $\mf(\no_g)$ by the $\mathbb{R_+}$-action given by scaling the transverse measure gives us the set of projective measured foliations $\pmf(\no_g)$.
The subsets $\mf^-(\no_g)$, $\mf^+(\no_g)$, and $\mf^+(\no_g, \QQ)$ are $\mathbb{R}$-invariant, and thus descend to their projective versions $\pmf^-(\no_g)$, $\pmf^+(\no_g)$, and $\pmf^+(\no_g, \QQ)$.
The set $\pmf(\no_g)$ is the boundary of the Teichm\"uller space of $\no_g$, and admits a continuous mapping class group action.
It is when considering the mapping class group action that we see differences between the orientable and the non-orientable case.

\begin{theorem}[Proposition 8.9 of \cite{gendulphe_whats_2017}]
  \label{thm:full-non-minimal}
  The action of $\mcg(\no_g)$ (for $g \geq 2$) on $\mf(\no_g)$ is not minimal.
  In fact, the action is not even topologically transitive.
\end{theorem}
\begin{remark}
  The proof of non minimality and topological non-transitivity follow from the fact that one can construct a $\mcg(\no_g)$-invariant continuous function on $\mf(\no_g)$.
  That is because starting with a foliation in $\mf^+(\no_g)$, it is impossible to approximate an element of $\mf^-(\no_g)$ since one does not have Dehn twists about one-sided curves.
\end{remark}

One can now consider subspaces of $\mf(\no_g)$ where the $\mcg(\no_g)$ action might be nicer. There are two natural subspaces: $\mf^+(\no_g)$, and $\mf^{-}(\no_g)$.
Danthony-Nogueira proved the following theorem about $\mf^{-}(\no_g)$ in \cite{ASENS_1990_4_23_3_469_0}.

\begin{theorem}[Theorem II of \cite{ASENS_1990_4_23_3_469_0}]
  \label{thm:DN90-2}
  $\mf^{-}(\no_g)$ is an open dense subset of $\mf(\no_g)$ of full Thurston measure.
\end{theorem}

\autoref{thm:full-non-minimal} and \autoref{thm:DN90-2} suggest that studying the $\mcg(\no_g)$ dynamics restricted to $\mf^{-}(\no_g)$ will be hard since one will not have minimality, or ergodicity with respect to any measure with full support.
In Section \ref{sec:lower-bound-limit-set}, we get a lower bound for the set on which $\mcg(\no_g)$ acts minimally.

\subsection{Limit sets of mapping class groups}
\label{sec:backgr-limit-sets}

The first results on limit sets of subgroups of mapping class groups were obtained by Masur for handlebody subgroups \cite{masur_1986}, and McCarthy-Papadopoulos for general mapping class subgroups \cite{McCarthy1989}.
They defined two distinct notions of limit sets; while they did not give distinct names to the two different definitions, we will do so, for the sake of clarity.
\begin{definition}[Dynamical limit set]
  Given a subgroup $\Gamma$ of the mapping class group, the dynamical limit set $\dynlim(\Gamma)$ is the minimal closed invariant subset of $\pmf$ under the action of $\Gamma$.
\end{definition}
\begin{definition}[Geometric limit set]
  Given a subgroup $\Gamma$ of the mapping class group, and a point $x$ in the Teichmüller space, its boundary orbit closure $\Lambda_{\mathrm{geo}, x}(\Gamma)$ is intersection of its orbit closure with the Thurston boundary, i.e. $\overline{\Gamma x} \cap \pmf$.
  The geometric limit set is the union of all boundary orbit closures, as we vary $x$ in the Teichmüller space, i.e. $\geolim(\Gamma) = \bigcup_{x \in \teich} \Lambda_{\mathrm{geo}, x}(\Gamma)$.
\end{definition}

\begin{remark}
  The specific family of subgroups considered by McCarthy-Papadopoulos were subgroups containing at least two non-commuting pseudo-Anosov mapping classes, in which case, the dynamical limit set is unique.
  Unless otherwise specified, we will only talk about mapping class subgroups which contain at least two non-commuting pseudo-Anosovs.
\end{remark}

Both of these definitions are natural generalizations of the limit sets of Fuchsian groups acting on $\HH^2$.
In the hyperbolic setting, the two notions coincide, but for mapping class subgroups, the dynamical limit set may be a proper subset of the geometric limit set.

For simple enough subgroups, one can explicitly work out $\dynlim(\Gamma)$ and $\geolim(\Gamma)$: for instance, when $\Gamma$ is the stabilizer of the Teichmüller disc associated to a Veech surface, $\dynlim(\Gamma)$ is the visual boundary of the Teichmüller disc, which by Veech dichotomy, only consists of either uniquely ergodic directions on the Veech surface, or the cylinder directions, where the coefficients on the cylinders are their moduli in the surface.
On the other hand, $\geolim(\Gamma)$ consists of all the points in $\dynlim(\Gamma)$, but it additionally contains all possible convex combinations of the cylinders appearing in $\dynlim(\Gamma)$ (see Section 2.1 of \cite{2007math......2034K}).

The gap between $\geolim$ and $\dynlim$ suggests the following expansion operation on subsets of $\pmf$
\begin{definition}[Expansion]
  Given a projective measured foliation $\lambda$, we define its expansion $\expansion(\lambda)$ to be the image in $\pmf$ of set of all non-zero measures invariant measures on the topological foliation associated to $\lambda$.
  Given a subset $\Lambda$, we define its expansion $\expansion(\Lambda)$ to be the union of expansions of the projective measured laminations contained in $\Lambda$.
\end{definition}
Observe that for a uniquely ergodic foliation $\lambda$, $\expansion(\lambda) = \{\lambda\}$, for a minimal but not uniquely ergodic $\lambda$, $\expansion(\lambda)$ is the convex hull of all the ergodic measures supported on the topological lamination associated to $\lambda$, and for a foliation with all periodic leaves, $\expansion(\lambda)$ consists of all foliations that can be obtained by assigning various weights to the core curves of the cylinders.

Going back to the example of the stabilizer of the Teichmüller disc of a Veech surface, we see that $\geolim(\Gamma) = \expansion(\dynlim(\Gamma))$.
One may ask if this is always the case.
\begin{question}
  Is $\geolim(\Gamma) = \expansion(\dynlim(\Gamma))$ for all $\Gamma$?
\end{question}

McCarthy-Papadopoulos also formulated an equivalent definition of $\dynlim(\Gamma)$, which is easier to work with in practice.

\begin{theorem}[Theorem 4.1 of \cite{McCarthy1989}]
  \label{thm:equivalence-of-limit-sets}
  $\dynlim(\Gamma)$ is the closure in $\pmf$ of the stable and unstable foliations of all the pseudo-Anosov mapping classes in $\Gamma$.
\end{theorem}
A corollary of this theorem is that $\Gamma$ does not act properly discontinuously on $\dynlim(\Gamma)$.
In fact, one can construct a domain of discontinuity for the $\Gamma$ action using $\dynlim(\Gamma)$.
\begin{definition}[Zero intersection set]
  Given a subset $\Lambda$ of $\pmf$, its zero intersection set $Z(\Lambda)$ is the following subset of $\pmf$
  \begin{align*}
    Z(\Lambda) \coloneqq \left\{ \lambda^{\prime} \mid i(\lambda, \lambda^{\prime}) = 0 \text{ for some $\lambda \in \Lambda$} \right\}
  \end{align*}
\end{definition}

% McCarthy-Papadopoulos also exhibit a domain of discontinuity for the $\Gamma$ action on $\pmf$.
\begin{theorem}[Theorem 6.16 of \cite{McCarthy1989}]
  The action of $\Gamma$ on $\pmf \setminus Z(\dynlim(\Gamma))$ is properly discontinuous.
\end{theorem}

It is not obvious that the action of $\Gamma$ on $Z(\dynlim(\Gamma))$ is not properly discontinuous.
We prove that is indeed the case when $\Gamma = \mcg(\no_g)$ in Section \ref{sec:domain-discontinuity}, but the general case is unknown.


\subsection*{List of notation}
Here we list the various symbols we will be using throughout the paper.
\begin{itemize}
\item[] $\os_g$: The compact orientable surface of genus $g$.
\item[] $\no_{d}$: The compact non-orientable surface of demigenus $d$.
\item[] $\iota$: The deck transformation of the orientation double cover of a non-orientable surface.
\item[] $\teich(S)$: The Teichm\"uller space of $S$.
\item[] $\systole(\no_d)$: The set of points in $\teich(\no_d)$ where no one-sided curve is shorter than
  $\varepsilon$.
\item[] $g_t$: The Teichm\"uller geodesic flow on quadratic differentials.
\item[] $\mcg(S)$: The mapping class group of $S$.
\item[] $\scc(S)$: The set of all simple closed curves on $S$.
\item[] $\mf(S)$: The space of measured foliations on $S$.
\item[] $\pmf(S)$: The space of projective measured foliations on $S$.
\item[] $\mf^+(\no_d)$, $\pmf^+(\no_d)$: The set of (projective) measured foliations on $\no_d$ containing
  no one-sided leaves.
\item[] $\mf^-(\no_d)$, $\pmf^-(\no_d)$: The set of (projective) measured foliations on $\no_d$ containing
  some one-sided leaf.
\item[] $\mf(S; \QQ)$, $\pmf(S; \QQ)$: The set of all (projective) weighted rational multicurves on $S$.
\item[] $\geolim(\Lambda)$: The geometric limit set of the discrete group $\Lambda$.
\item[] $\dynlim(\Lambda)$: The dynamical limit set of the discrete group $\Lambda$.
\item[] $\lhyp(M, \gamma)$: The hyperbolic length of $\gamma$ with respect to the hyperbolic structure on $M \in \teich(S)$. We will suppress $M$ when it is clear from context.
\item[] $\lflat(q, \gamma)$: The flat length of $\gamma$ with respect to the flat structure given by the DQD $q$. We will suppress $q$ when it is clear from context.
\item[] $o(f)$: A function $g$ is $o(f)$ if $\frac{g}{f}$ goes to $0$ as the value of $f$ goes to $\infty$.
\item[] $\mu_{c}$: The probability measure on a transverse arc given by the closed curve $c$.
\end{itemize}

\section{Lower bound for the limit set}
\label{sec:lower-bound-limit-set}

A natural lower bound for $\dynlim(\no_g)$ is the closure of the set of rational two-sided multicurves $\pmf^+(\no_g, \QQ)$.
For any $\lambda \in \pmf^+(\no_g, \QQ)$, and any psuedo-Anosov $\gamma$, conjugating $\gamma$ with large enough powers of the Dehn multi-twist given by $\lambda$ gives us a sequence of pseudo-Anosov maps whose stable foliation approaches $\lambda$, which shows that $\dynlim(\no_g)$ must contain $\lambda$.
Note that the same argument does not work if $\lambda \in \pmf^-(\no_g, \QQ)$, since one cannot Dehn twist about one-sided curves.
In Section \ref{sec:upper-bound-limit-set}, we show that the geometric limit set is indeed contained in the complement of $\pmf^-(\no_g)$.

In \cite{gendulphe_whats_2017}, Gendulphe made the following conjecture about $\overline{\pmf^+(\no_g, \QQ)}$.
\begin{conjecture}[Conjecture 9.1 of \cite{gendulphe_whats_2017}]
  \label{conj:gendulphe-1}
  For $g \geq 4$, $\pmf^+(\no_g) = \overline{\pmf^+(\no_g, \QQ)}$.
\end{conjecture}
We prove a slightly weaker version of the above conjecture, by describing a subset of the foliations that can be approximated by multicurves in $\pmf^+(\no_g, \QQ)$.
To state the theorem, we need to define what it means for a foliation to be orientable.
\begin{definition}[Orientable foliation]
  A foliation $\lambda$ is said to be orientable if there exists a small enough transverse arc $\eta$ such that any leaf exiting from one side of $\eta$ always returns from the other side.
\end{definition}
In the setting of orientable surfaces, the vertical foliations of translation surfaces are orientable, while there are some directions in half-translation surfaces where the foliation is non-orientable.
There exist similar examples of orientable and non-orientable foliations on non-orientable surfaces.

Having defined the notion of orientable foliations, we can state the main theorem of this section.
\begin{theorem}
  \label{thm:rational-approximation}
  For $g \geq 3$, a foliation $\lambda \in \pmf^+(\no_g)$ can be approximated by foliations in $\pmf^+(\no_g, \QQ)$ if all the minimal components $\lambda_j$ of $\lambda$ satisfy one of the following criteria.
  \begin{enumerate}[(i)]
  \item $\lambda_j$ is periodic.
  \item $\lambda_j$ is orientable.
  \item $\lambda_j$ is uniquely ergodic.
  \end{enumerate}
  Furthermore, if $\lambda_j$ is minimal, but not uniquely ergodic, there exists some other foliation $\lambda_j^{\prime}$ supported on the same topological foliation as $\lambda_j$ that can be approximated by elements of $\pmf^+(\no_g, \QQ)$.
\end{theorem}

To prove \autoref{thm:rational-approximation}, we will first need to describe a method of resolving intersections of curves.
Given two two-sided simple curves $\gamma_1$ and $\gamma_2$, their union will not be a simple curve if $i(\gamma_1, \gamma_2) > 0$.
However, at each point of intersection, there exist two possible surgeries that resolve the intersection.
% \begin{figure}[h]
%   \centering
%   \includegraphics{example-image-a}
%   \caption{Two ways an intersection can be resolved.}
%   \label{fig:surgery-types}
% \end{figure}
Resolving all the intersections by choosing surgeries at each of the points results in a simple multicurve (i.e. it may have more than one component).
It is not necessarily the case that the resulting multicurve has two-sided components either.
However, if it is the case that the resulting multicurve is a curve, then it will be a two-sided curve: that is because both $\gamma_1$ and $\gamma_2$ pass through an even number of crosscaps, and if the resolution of $\gamma_1 \cup \gamma_2$ has only one component, it must also pass through an even number of crosscaps.
We prove that there always exists a choice of surgery that results in a simple curve.
\begin{lemma}
  \label{lem:valid-surgery}
  Let $\gamma_1$ and $\gamma_2$ be two-sided curves on $\no_g$ with $i(\gamma_1, \gamma_2) > 0$.
  For each intersection point of $\gamma_1$ and $\gamma_2$, there exists a choice of surgery to resolve the intersection such that the multicurve obtained by resolving all of the singularities contains only one component.
\end{lemma}
\begin{proof}
  We begin by turning the union of $\gamma_1$ and $\gamma_2$ into a non-simple curve.
  Pick any intersection point, and resolve the singularity (the choice of resolution does not matter).
  It is easily verified that the resulting object can be parameterized as a map from $S^1$ into the surface.
  If $i(\gamma_1, \gamma_2) = 1$, we are done, since we have a simple curve.
  If $i(\gamma_1, \gamma_2) > 1$, we have a non-simple curve.
  At each intersection point, we resolve the singularity such that the strands corresponding to directions flowing into the singularity are glued together, and the strands corresponding to directions flowing out of the singularity are glued together (see \autoref{fig:good-surgery}).
  \begin{figure}[h]
    \centering
    \incfig[0.8]{good-surgery}
    \caption{Surgery that results in a connected curve.}
    \label{fig:good-surgery}
  \end{figure}
Again, it is easy to see that this surgery results in a connected curve, and after resolving all singularities in this manner, we have a simple curve.
\end{proof}

We will call the resulting simple curve $\kappa$.
For all that follows, we will pick an arbitrary hyperbolic metric on our surface.
We want the geodesic representative of $\kappa$ to have approximately the same intersection numbers with all curves as the $\gamma_1 \cup \gamma_2$ did.
For that to happen, it is important the geodesic tightening of $\kappa$ fellow travels with both $\gamma_1$ and $\gamma_2$ for a large fraction of their lengths.
The next lemma describes the conditions for that to happen.
\begin{lemma}
  \label{lem:geodesic-tightening}
  Let $\alpha$, $\beta$ and $\gamma$ be three sides of a geodesic triangle in $\HH^2$.
  Let $c_\alpha$ and $c_{\beta}$ be points on $\alpha$ and $\beta$ equidistant from the intersection of $\alpha$ and $\beta$ such that $d(\alpha, \beta) = 4 \delta$, where $\delta$ is the Gromov hyperbolicity constant of $\HH^2$.
  Then the broken geodesic formed by going from the intersection of $\alpha$ and $\gamma$ to $c_{\alpha}$ to $c_{\beta}$ to the intersection of $\gamma$ and $\beta$ is in a $2\delta$-neighbourhood of $\gamma$ and $\gamma$ is in a $4\delta$-neighbourhood of this broken geodesic.
\end{lemma}
\begin{proof}
  Denote the segment between the intersection point of $\alpha$ and $\gamma$ and $c_{\alpha}$ by $\alpha^{\prime}$, and analogous segment on $\beta$ by $\beta^{\prime}$.
  Denote the geodesic segment between $c_{\alpha}$ and $c_{\beta}$ by $\kappa$.
  The segments $\gamma$, $\alpha^{\prime}$, $\kappa$ and $\beta^{\prime}$ form a geodesic quadrilateral.
  By Gromov hyperbolicity, we have that $\gamma$ is in a $2\delta$-neighbourhood of the broken geodesic.
  Since the distance between $\alpha^{\prime}$ and $\beta^{\prime}$ is more than $4 \delta$, the there must be some point in $\kappa$ that is within $2 \delta$ distance of $\gamma$.
  To see that the opposite inclusion also holds, note that $\beta^{\prime}$ is in a $2\delta$-neighbourhood of $\gamma$, $\alpha^{\prime}$ and $\kappa$.
  But $\alpha^{\prime}$ is more than $4 \delta$ distance away from $\beta^{\prime}$, and thus $\beta^{\prime}$ must be in a $2\delta$ neighbourhood of $\gamma$ and $\kappa$.
  But since $\kappa$ itself is within $2\delta$ distance of $\gamma$, the $4\delta$-neighbourhood of $\gamma$ contains $\beta^{\prime}$.
  The same argument also shows containment of $\alpha^{\prime}$ and $\kappa$, proving the lemma.
  See \autoref{fig:segment-together} for a picture of the setup.
  \begin{figure}[h]
    \centering
    \incfig[0.8]{segment-together}
    \caption{Resolved segments that get shorter under geodesic tightening.}
    \label{fig:segment-together}
  \end{figure}
\end{proof}
To use \autoref{lem:geodesic-tightening} in our situation, we will additionally need to argue that the lengths of $\alpha^{\prime}$ and $\beta^{\prime}$ are almost the entire length of $\alpha$ and $\beta$ respectively.
That will show that the $\kappa$ and $\gamma_1 \cup \gamma_2$ have approximately the same intersection numbers with any simple closed curve (after scaling appropriately).

Given a sequence of curves $\{\gamma_j\}$ converging to a projective measured foliation in $\pmf$, we can also study their convergence in $\mf$, after scaling by an appropriate constant.
The next lemma shows that it suffices to scale by their hyperbolic length.
\begin{lemma}
  \label{lem:scaling-lemma}
  If $\{\gamma_j\}$ converges to a projective measured foliation in $\pmf$, then $\displaystyle \frac{\gamma_j}{\lhyp(\gamma_j)}$ converges in $\mf$, possibly after passing to a subsequence.
\end{lemma}
\begin{proof}
  We claim that for any curve $\xi$, $i(\gamma_j, \xi) \leq c \cdot \lhyp(\gamma_j)$, where the constant $c$ only depends on the hyperbolic metric.
  This is because the $\lhyp(\gamma_j)$ is bounded below by $i(\gamma_j, \xi) \cdot o_\xi$, where $o_\xi$ is the length of the shortest orthogeodesic arc beginning and ending at $\xi$.
  As $\xi$ varies among all simple closed curves, $o_\xi$ is bounded below by an absolute constant: this follows from hyperbolic trigonometry (see \cite{thurston1979geometry}).
  That proves the claim.

  In particular, we have that intersection with $\displaystyle \frac{\gamma_j}{\lhyp(\gamma_j)}$ is a bounded transverse measure.
  By compactness of bounded measures, we have convergence after passing to a subsequence.
\end{proof}

\begin{lemma}
  \label{lem:convex-combinations}
  Let $\gamma_1$ and $\gamma_2$ be two measured foliations supported on the same topological foliation such that the transverse measures induced by $\gamma_1$ and $\gamma_2$ are mutually singular.
  Furthermore, suppose both $\gamma_1$ and $\gamma_2$ are approximable by simple two-sided curves.
  Then any convex combination $c_1 \gamma_1 + c_2\gamma_2$ is also approximable by simple two-sided curves.
\end{lemma}

  % \todo[inline]{Make proof clearer by emphasizing on the importance of the multiplicity $m_j$.}
\begin{proof}
  Let $\{\gamma_{1j}\}$ and $\{ \gamma_{2j}\}$ be sequences of simple two-sided curves converging to $\gamma_1$ and $\gamma_2$ in $\pmf$.
  We can assume without loss of generality that $i(\gamma_{1i}, \gamma_{2j^{\prime}}) > 0$ for all $j$ and $j^{\prime}$ (to see this, Dehn twist one of the curves by a short curve that intersects both of them).
  We can also assume using \autoref{lem:scaling-lemma} that normalizing these curves by their hyperbolic length converges in $\mf$.
  We pass to a subsequence of $\gamma_{2j}$ such that after passing to the subsequence,
  the following holds.
  \begin{align}
    \label{eq:1}
    \lim_{i \to \infty} \frac{\lhyp(\gamma_{2j})}{\lhyp(\gamma_{1j})} = \infty
  \end{align}
  We take the union of $m_j$ parallel copies of $\gamma_{1j}$ and $1$ copy of $\gamma_{2j}$, where we pick $m_j$ such the fraction $\displaystyle \frac{m_j \cdot \lhyp(\gamma_{1j})}{\lhyp(\gamma_{2j})}$ is as close to $\displaystyle \frac{c_1}{c_2}$ as possible.
  Equation \eqref{eq:1} will ensure that the ratio of the lengths (with multiplicities) will approach $\displaystyle \frac{c_1}{c_2}$ as $i$ goes to $\infty$.

  We now resolve the intersections of the $m_j$ copies of $\gamma_{1j}$ and $\gamma_{2j}$ so that the resulting multicurve has only one component.
  If $m_j = 1$, \autoref{lem:valid-surgery} tells us that there is such a resolution of intersections.
  If $m_j > 1$, we resolve the intersections the same way as in the case of $m_j = 1$, but also glue together the strands of $\gamma_{1j}$ as shown in \autoref{fig:parallel-surgery-modification}.
  \begin{figure}[h]
    \centering
    \incfig[1]{parallel-surgery-modification}
    \caption{Modifying the resolution when $m_j > 1$.}
    \label{fig:parallel-surgery-modification}
  \end{figure}

  Denote the resolved curve by $\kappa_j$.
  To show that $\kappa_j$ converges to $c_1 \gamma_1 + c_2 \gamma_2$ in $\pmf$, we need to show that the following limit holds for all simple closed curves $\xi$.
  \begin{align}
    \label{eq:2}
    \lim_{j \to \infty} \frac{i(\kappa_j, \xi)}{m_j \cdot \lhyp(\gamma_{1j}) + \lhyp(\gamma_{2j})}
    = \lim_{j \to \infty} \left( c_1 \frac{i(\gamma_{1j}, \xi)}{\lhyp(\gamma_{1j})} + c_2 \frac{i(\gamma_{2j}, \xi)}{\lhyp(\gamma_{2j})} \right)
  \end{align}
  The equality replaced with an inequality in one direction is easy to see: namely, the left hand side is less than or equal to the right hand side.
  This is because resolving intersections can only reduce the intersection number with any given simple closed curve.
  Thus, we only need to prove the other inequality to prove the lemma.
  \begin{align}
    \label{eq:3}
    \lim_{j \to \infty} \frac{i(\kappa_j, \xi)}{m_j \cdot \lhyp(\gamma_{1j}) + \lhyp(\gamma_{2j})}
    \geq \lim_{j \to \infty} \left( c_1 \frac{i(\gamma_{1j}, \xi)}{\lhyp(\gamma_{1j})} + c_2 \frac{i(\gamma_{2j}, \xi)}{\lhyp(\gamma_{2j})} \right)
  \end{align}
  Let $\gamma_{1j}^{\prime}$ and $\gamma_{2j}^{\prime}$ denote the segments of lifts to $\HH^2$ of $\gamma_{1j}$ and $\gamma_{2j}$ that fellow travel within distance $4\delta$ after intersecting, and denote their complements in $\gamma_{1j}$ and $\gamma_{2j}$ by $\gamma_{1j}^{\prime \prime}$ and $\gamma_{2j}^{\prime \prime}$.
  \autoref{lem:geodesic-tightening} tells us that the geodesic representative of $\kappa_j$ will be within bounded distance of $\gamma_{1j}^{\prime \prime}$ and $\gamma_{2j}^{\prime \prime}$.
  For $j$ large enough, we can ensure that the angle of intersection of $\xi$ and $\gamma_{1j}$ and $\gamma_{2j}$ is larger than $\varepsilon_\xi$ for some $\varepsilon_\xi > 0$, where the angle $\varepsilon_\xi$ depends on $\xi$.
  That means we can make $\gamma_{1j}^{\prime}$ and $\gamma_{2j}^{\prime}$ longer by a fixed amount, depending on $\varepsilon_\xi$ (and make $\gamma_{1j}^{\prime \prime}$ and $\gamma_{2j}^{\prime \prime}$ correspondingly shorter) such that whenever $\xi$ intersects $\gamma_{1j}^{\prime \prime}$ or $\gamma_{2j}^{\prime \prime}$, it also intersects $\kappa_j$.
  This fact gives us a lower bound on the intersection number with $\kappa_j$.
  \begin{align}
    \label{eq:4}
    i(\kappa_j, \xi) \geq m_j \cdot i(\gamma_{1j}^{\prime \prime}, \xi) + i(\gamma_{2j}^{\prime \prime}, \xi)
  \end{align}
  We express the right hand side as the intersection number with $\gamma_{1j}$ and $\gamma_{2j}$ minus an error term $\err{\xi, j}$.
  \begin{align*}
    i(\kappa_j, \xi) &\geq \left( m_j \cdot  i(\gamma_{1j}, \xi) + i(\gamma_{2j}, \xi)  \right)
    - \left( m_j \cdot  i(\gamma_{1j}^{\prime}, \xi) + i(\gamma_{2j}^{\prime}, \xi)  \right) \\
    &= \left( m_j \cdot  i(\gamma_{1j}, \xi) + i(\gamma_{2j}, \xi)  \right) - \err{\xi, j}
  \end{align*}
  We claim that $\err{\xi, j}$ is $o(m_j \cdot  i(\gamma_{1j}, \xi) + i(\gamma_{2j}, \xi))$: this will prove inequality \eqref{eq:3} and thus the lemma.

  To begin with, we show that the lengths of $\gamma_{1j}^{\prime}$ and $\gamma_{2j}^{\prime}$ are $o(\lhyp(\gamma_{1j}))$ and $o(\lhyp(\gamma_{2j}))$.
  Suppose it was not the case, and $\displaystyle \frac{\lhyp(\gamma_{1j}^{\prime})}{\lhyp(\gamma_{1j})} \geq k_1$, for some positive constant $k_1$.
  Then any short transverse arc that only intersected $\gamma_{1j}^{\prime}$ would get assigned a positive measure as $j$ went to $\infty$.
  Since the lengths of $m_j \cdot \gamma_{1j}$ and $\gamma_{2j}$ approach a fixed ratio, we would have that $\displaystyle \frac{\lhyp(\gamma_{2j}^{\prime})}{\lhyp(\gamma_{2j})} \geq k_2$ for some other positive constant $k_2$.
  Thus the same short arc would also get assigned a positive measure by $\gamma_{2j}$ as $j$ went to $\infty$.
  This means that the transverse measure given $\gamma_1$ decomposes as $\gamma_1{\prime} + \gamma_1^{\prime \prime}$, where $\gamma_1^{\prime}$ is the limit of the transverse measures given by $\gamma_{1j}^{\prime}$ and $\gamma_1^{\prime \prime}$ is the limit of the transverse measures given by the limit of $\gamma_{1j}^{\prime \prime}$.
  We get a similar decomposition of $\gamma_2$ into $\gamma_2^{\prime} + \gamma_2^{\prime \prime}$.
  Our argument shows that $\gamma_{1}^{\prime}$ and $\gamma_2^{\prime}$ are absolutely continuous with respect to each other.
  But this violates mutual singularity of the measures $\gamma_1$ and $\gamma_2$, and therefore, the lengths of $\gamma_{1j}^{\prime}$ and $\gamma_{2j}^{\prime}$ must be $o(\lhyp(\gamma_{1j}))$ and $o(\lhyp(\gamma_{2j}))$ respectively.
  We sum up the result of this argument in the following inequalities.
  \begin{align}
    \label{eq:5}
    \lim_{j \to \infty} \frac{\lhyp(\gamma_{1j}^{\prime})}{\lhyp(\gamma_{1j})} &= 0 \\
    \label{eq:6}
    \lim_{j \to \infty} \frac{\lhyp(\gamma_{2j}^{\prime})}{\lhyp(\gamma_{2j})} &= 0 \\
  \end{align}

  The bound on the length of $\gamma_{1j}^{\prime}$ and $\gamma_{2j}^{\prime}$ also gives a bound on $i(\xi, \gamma_{1j}^{\prime})$ and $i(\xi, \gamma_{2j}^{\prime})$.
  This follows from the fact that the intersection number of any arc with a fixed curve is bounded above by a constant times the length of the arc, where the constant depends on the curve.
  \begin{align}
    \label{eq:7}
    i(\xi, \gamma_{1j}^{\prime}) &\leq c_\xi \lhyp(\gamma_{1j}^{\prime}) \\
    \label{eq:8}
    i(\xi, \gamma_{2j}^{\prime}) &\leq c_\xi \lhyp(\gamma_{2j}^{\prime}) \\
  \end{align}

  Finally, \autoref{lem:scaling-lemma} gives us an upper bound for the ratio between intersection number and length of $\gamma_{1j}$ and $\gamma_{2j}$.
  \begin{align}
    \label{eq:9}
    \frac{\lhyp(\gamma_{1j})}{i(\xi, \gamma_{1j})} &\leq k \\
    \label{eq:10}
    \frac{\lhyp(\gamma_{2j})}{i(\xi, \gamma_{2j})} &\leq k \\
  \end{align}
  Multiplying \eqref{eq:5}, \eqref{eq:7}, and \eqref{eq:9}, and \eqref{eq:6}, \eqref{eq:8}, and \eqref{eq:10} proves the claim about the error term, and therefore the lemma.
\end{proof}

To state our next lemma, we need to define the \emph{orbit measure} associated to simple curve, and define what it means for an orbit measure to be \emph{almost invariant}.
Consider an arc $\eta$ transverse to a measured foliation $\lambda$.
We assign one of the sides of $\eta$ to be the ``up'' direction, and the other side to be the ``down'' direction.
This lets us define the first return map to $T$.
\begin{definition}[First return map]
  The first return map $T$ maps a point $p \in \eta$ to the point obtained by flowing along the foliation in the ``up'' direction until the flow intersects $\eta$ again.
  The point of intersection is defined to be $T(p)$.
  If the flow terminates at a singularity, $T(p)$ is left undefined: there are only countable many points in $\eta$ such that this happens.
\end{definition}
Since $\lambda$ is a measured foliation, it defines a measure on $\eta$: we can assume that it is a probability measure.
It follows from the definition of transverse measures that the measure is $T$-invariant.
It is a classical result of Katok \cite{zbMATH03467479}  and Veech \cite{Veech1978} that the set of $T$-invariant probability measures is a finite dimensional simplex contained in the Banach space of bounded signed measures on $\eta$.
Given an orbit of a point $p$ under the $T$-action of length $L$, we can construct a probability measure on $\eta$, called the orbit measure of $p$.
\begin{definition}[Orbit measure]
  The orbit measure of length $L$ associated to the point $p$ is the following probability measure on $\eta$.
  \begin{align*}
    \mu_{p, L} \coloneqq \frac{1}{L} \sum_{i=0}^{L-1} \delta_{T^i(p)}
  \end{align*}
  Here, $\delta_{x}$ is the Dirac delta measure at the point $x$.
\end{definition}
One might expect that if a point $p$ equidistributes, then a long orbit measure starting at $p$ will be ``close'' to an invariant measure.
We formalize this notion by metrizing the Banach space of signed finite measures on $\eta$.
\begin{definition}[Lèvy-Prokhorov metric]
  Define $\norm{\cdot}_{\mathrm{BL}}$ denote the bounded Lipschitz norm on the space of continuous functions on $\eta$.
  \begin{align*}
    \norm{f}_{\mathrm{BL}} \coloneqq \norm{f}_{\infty} + \sup_{x \neq y} \frac{\left| f(x) - f(y) \right|}{\left| x - y \right|}
  \end{align*}
\end{definition}
Then the Lèvy-Prokhorov distance $d_{\mathrm{LP}}$ between the probability measures $\mu_1$ and $\mu_2$ is defined to be the following.
\begin{align*}
  d_{\mathrm{LP}}(\mu_1, \mu_2) \coloneqq \sup_{\norm{f}_{\mathrm{BL}} \leq 1} \int f (\dd \mu_1 - \dd \mu_2)
\end{align*}
Using the Lèvy-Prokhorov metric, we can define what it means for a probability measure to be $\varepsilon$-almost $T$-invariant.
\begin{definition}[$\varepsilon$-almost $T$-invariance]
  A measure $\mu$ is $\varepsilon$-almost $T$-invariant if $d_{\mathrm{LP}}(\mu, T \mu) \leq \varepsilon$. Here $T \mu$ is the pushforward of $\mu$ under $T$.
\end{definition}
We state the following easy fact about orbit measures without proof.
\begin{fact}
  An orbit measure of length $L$ is $\frac{2}{L}$-almost $T$-invariant.
\end{fact}
The following lemma shows that a long orbit measure is close to an invariant measure.
\begin{lemma}
  \label{lem:long-orbit-is-almost-invariant}
  For any $\varepsilon > 0$, there exists an $L$ large enough such that any orbit measure longer than $L$ is within distance $\varepsilon$ of an invariant measure.
\end{lemma}
\begin{proof}
  Let $M_L$ denote the set of all orbit measures of length greater than or equal to $L$.
  The set $M_L$ is compact, because it is a closed subset of a compact set, and we have that $\bigcap_{L=1}^{\infty} M_L$ is the set of invariant measures.
  By compactness, we have that for some large enough $L$, $M_L$ must be in a $\varepsilon$-neighbourhood of the set of invariant measures.
\end{proof}
We now prove a lemma that gives us a criterion for deducing when a long orbit measure is close to an ergodic measure.
\begin{lemma}
  \label{lem:convexity-argument}
  Let $\{n_i\}$ be a sequence of positive integers, and let $\{p_{ij}\}$ and $\{L_{ij}\}$ be points in $\eta$ and positive integers respectively, where $1 \leq j \leq n_i$ and $\min_j L_{ij}$ goes to $\infty$ as $i$ goes to $\infty$.
  Consider the following sequence of probability measures, indexed by $i$.
  \begin{align*}
    \mu_i \coloneqq \frac{\sum_{j=1}^{n_i} L_{ij} \cdot \mu_{p_{ij}, L_{ij}}}{\sum_{j=1}^{n_i} L_{ij}}
  \end{align*}
  If the sequence $\{\mu_i\}$ converges to an ergodic measure $\nu$, then there exists a subsequence of the orbit measures $\mu_{p_{ij}, L_{ij}}$ also converging to $\nu$.
\end{lemma}
\begin{proof}
  Suppose for the sake of a contradiction that no subsequence of $\mu_{p_{ij}, L_{ij}}$ converged to $\nu$.
  That would mean there exists a small enough $\varepsilon > 0$ and a large enough $i_0$ such that for all $i > i_0$, the measures $\mu_{p_{ij}, L_{ij}}$ are more than distance $\varepsilon$ from $\nu$.
  Since $\min_{j} L_{ij}$ goes to $\infty$, there exists some other large enough $i_1 > i_0$ such that for all $i > i_1$, $\mu_{p_{ij}, L_{ij}}$ is within distance $\frac{\varepsilon}{k}$ of the simplex of invariant probability measures, where $k$ is a large integer we will pick later: this is a consequence of \autoref{lem:long-orbit-is-almost-invariant}.
  Using this, we decompose $\mu_{p_{ij}, L_{ij}}$ as the sum of an invariant measure $\iota_{ij}$ and a signed measure $e_{ij}$, such that $d_{\mathrm{LP}}(0, e_{ij}) \leq \frac{\varepsilon}{k}$.
  \begin{align*}
    \mu_{p_{ij}, L_{ij}} = \iota_{ij} + e_{ij}
  \end{align*}
  Observe that the weighted average of $\mu_{p_{ij}, L_{ij}}$ will differ from the weighted average of $\iota_{ij}$ by at most $\frac{\varepsilon}{k}$.
  Also note that all the invariant measures $\iota_{ij}$ are distance at least $\varepsilon - \frac{\varepsilon}{k}$ from $\nu$.
  Since $\nu$ is the vertex of a finite-dimensional convex set, any weighted average of the $\iota_{ij}$ must be at least distance $\frac{\varepsilon}{k^{\prime}}$ from $\nu$, where $k^{\prime}$ is a positive number depending on the geometry of the convex set of invariant probability measures.
  By picking $k > 2k^{\prime}$ we can ensure that any weighted average of the $\mu_{p_{ij}, L_{ij}}$ must be at least distance $\frac{\varepsilon}{2k^{\prime}}$ from $\nu$.
  But this would contradict our hypothesis that the measures $\mu_i$ converge to $\nu$.
  That would that there exists some subsequence of $\mu_{p_{ij}, L_{ij}}$ that converges to $\nu$, which proves the lemma.
\end{proof}

We now consider a tricky special case we are forced to reckon with when approximating ergodic foliations with simple closed curves.
Let $\lambda$ be a minimal foliation and $\eta$ a transverse arc.
Let $q$ be a DQD on $\no_g$ such that $\lambda$ is the vertical foliation, and $\eta$ is a horizontal arc of flat length $1$.
Let $a$ be a leaf starting at the left endpoint $p$ of $\eta$ and going up, and suppose it comes back from the top intersecting $\eta$ a point $q$ which is distance $\varepsilon \ll 1$ from the left endpoint (possibly after intersecting $\eta$ several times).
Let $c$ be the simple closed curve obtained by concatenating $a$ and the horizontal arc joining $p$ and $q$.
% \begin{figure}[h]
%   \centering
%   \includegraphics{example-image-a}
%   \caption{Closing up leaves that start and begin on the same side of the arc.}
%   \label{fig:bad-situation}
% \end{figure}
The curve $c$ is clearly not close to the flat or hyperbolic geodesic representative of its homotopy class.
One obvious way to make $c$ shorter without changing its homotopy type is to flow the arc joining $p$ and $q$ along the foliation until it encounters a singularity, at which point we stop.
We label the new endpoints $p^{\prime}$ and $q^{\prime}$, and define a new closed curve $c^{\prime}$ by starting at $p^{\prime}$, flowing along the vertical foliation until $q^{\prime}$ and then flowing horizontally back to $p^{\prime}$.
We consider the transverse probability measures on $\eta$ given by intersection with $c$, $c^{\prime}$, and their hyperbolic geodesic representative, which we denote $c^{\prime \prime}$.
We denote the associated transverse probability measures on $\eta$ by $\mu_c$, $\mu_{c^{\prime}}$, and $\mu_{c^{\prime \prime}}$ respectively.
The following lemma states that the $\mu_{c^{\prime}}$ and $\mu_{c^{\prime \prime}}$ are approximately equal.

\begin{lemma}
  \label{lem:flat-length-tightening}
  As $\varepsilon$ approaches $0$, $d_{\mathrm{LP}}(\mu_{c^{\prime}}, \mu_{c^{\prime \prime}})$ approaches $0$.
\end{lemma}
\begin{proof}
  Note that as $\varepsilon$ goes to $0$, the hyperbolic lengths of $c^{\prime}$ and $c^{\prime \prime}$ go to $\infty$.
  If we show that $c^{\prime}$ and $c^{\prime \prime}$ stay within a bounded distance of each other, independent of $\varepsilon$, the measures $\mu_{c^{\prime}}$ and $\mu_{c^{\prime \prime}}$ converge to the same measures, since the intersection points in $\eta$ with $c^{\prime}$ and $c^{\prime \prime}$ will get closer and closer, by elementary hyperbolic geometry.

  Observe that the closed curve $c^{\prime}$ is almost a hyperbolic geodesic: namely the segment of it that follows the foliation $\lambda$: the only possibility is that the arc joining $p^{\prime}$ and $q^{\prime}$ results in the closed curve shortening significantly.
  However, note that the segment of $c^{\prime}$ that follows the foliation approaches the singularity from two different prongs which have a positive angle between them.
  Joining the ends with a short geodesic gives us a segment obtained by concatenating three geodesic arcs.
  Gromov hyperbolicity tells us that the geodesic representative of such a segment will be within a bounded distance of the segment: i.e. the geodesic representative $c^{\prime \prime}$ will be within a bounded distance of $c^{\prime}$ (see \autoref{fig:geodesic-tightening}).
  \begin{figure}[h]
    \centering
    \incfig[1]{geodesic-tightening}
    \caption{The curves $c^{\prime}$ and $c^{\prime \prime}$ are within bounded distance because all the prongs of the singularity have a positive angle between them.}
    \label{fig:geodesic-tightening}
  \end{figure}
  This proves the lemma.
\end{proof}
The next lemma links the ratio of the flat lengths of $c$ and $c^{\prime}$ to the measures $\mu_c$ and $\mu_{c^{\prime}}$.
\begin{lemma}
  \label{lem:absolute-continuity}
  Consider a sequence of $\varepsilon$ going to $0$ such that $\displaystyle \frac{\lflat(c^{\prime})}{\lflat(c)} \geq \nu > 0$ and both $\mu_c$ and $\mu_{c^{\prime}}$ converge.
  Then $\lim_{\varepsilon \to 0} \mu_{c^{\prime}}$ is absolutely continuous with respect to $\lim_{\varepsilon \to 0} \mu_c$.
\end{lemma}
\begin{proof}
  Consider a sub-interval $\eta^{\prime}$ of $\eta$ such that $\lim_{\varepsilon \to 0}\mu_c(\eta^{\prime}) = 0$.
  We need to show that the same holds for the limit of the measures $\mu_{c^{\prime}}$.
  Let $i(c, \eta)$ and $i(c, \eta^{\prime})$ be the number of times $c$ intersects $\eta$ and $\eta^{\prime}$, and let $i(c^{\prime}, \eta)$ and $i(c^{\prime}, \eta^{\prime})$ be defined similarly.
  Since we are working with a fixed DQD, the intersection number of a vertical arc with $\eta$ is bounded above and below by the flat length of the vertical arc times a positive constant depending only on $\eta$ (the positive constants are the maximum and minimum heights of the rectangles obtained when representing the flat structure as a zippered rectangle).
  \begin{align}
    \label{eq:11}
    \alpha_{\eta} \lflat(c) \leq &i(c, \eta) \leq \beta_{\eta} \lflat(c) \\
    \label{eq:12}
    \alpha_{\eta} \lflat(c^{\prime}) \leq &i(c^{\prime}, \eta) \leq \beta_{\eta} \lflat(c^{\prime})
  \end{align}
  Using the inequality in the hypothesis, and \eqref{eq:11} and \eqref{eq:12}, we get the following chain of inequalities.
  \begin{align*}
    \frac{i(c^{\prime}, \eta^{\prime})}{i(c^{\prime}, \eta)}
    &\leq \frac{i(c, \eta^{\prime})}{i(c^{\prime}, \eta)} \\
    &\leq \frac{1}{\alpha_{\eta}} \cdot \frac{i(c, \eta^{\prime})}{\lflat(c^{\prime})} \\
    &\leq \frac{1}{\alpha_{\eta} \nu} \cdot \frac{i(c, \eta^{\prime})}{\lflat(c)} \\
    &\leq \frac{\beta_{\eta}}{\alpha_{\eta} \nu} \cdot \frac{i(c, \eta^{\prime})}{i(c, \eta)}
  \end{align*}
  The right hand side goes to $0$ as $\varepsilon$ goes to $0$, proving absolute continuity of the limiting measure.
\end{proof}

We now have everything we need to prove \autoref{thm:rational-approximation}.
% \todo[inline]{Make the proof clearer.}
\begin{proof}[Proof of \autoref{thm:rational-approximation}]
  If a minimal component $\lambda_j$ is periodic, then the proof is straightforward.
  Since $\lambda$ contains no one-sided component, the core curve of $\lambda_j$ must be two-sided, possibly with an irrational coefficient.
  Approximating the core curve with rational coefficients proves the result in case (i).

  In case (ii), we have that $\lambda_j$ is not periodic, but a minimal orientable foliation.
  If we know that all the ergodic measures supported on the underlying topological foliation can be approximated by simple two-sided curves, we can approximate any convex combination of those measures, using \autoref{lem:convex-combinations}, since all the ergodic measures are mutually singular.
  It will therefore suffice to deal with the case that $\lambda_j$ is ergodic.

  Pick an arc $\eta_0$ transverse to $\lambda_j$ such that the leaf passing through the left endpoint $p_0$ of $\eta_0$ equidistributes with respect to the ergodic transverse measure of $\lambda_j$.
  We can find such a leaf because almost every leaf equidistributes with respect to the ergodic measure.
  We now inductively define a sequence of points $\{p_i\}$, sequence of sub-intervals $\eta_i$, and a sequence of segments $\{a_i\}$ of the leaf passing through $p_0$.
  Let $p_1$ be the first return of the leaf going up through $p_0$ to the interval $\eta_0$.
  Define the sub-interval $\eta_1$ to be the sub-interval whose left endpoint is $p_0$ and right endpoint is $p_1$.
  Let $a_1$ be the segment of the leaf starting at $p_0$ and ending at $p_1$.
  Given a point $p_i$, define $p_{i+1}$ to be the first return to the interval $\eta_{i}$, $\eta_{i+1}$ to be the interval whose left endpoint is $p_0$ and right endpoint is $p_{i+1}$, and $a_{i+1}$ to be the segment of the leaf starting at $p_i$ and ending at $p_{i+1}$.
  % Let $A_{i}$ be the concatenation of the leaf segments $\{a_1, \ldots, a_i\}$.

  Since we have assumed $\lambda_j$ is an orientable foliation, we have that the leaf we are working with always enters $\eta_0$ from the bottom, and exits from the top.
  If we pick $\eta_0$ to be small enough, we can pick a local orientation, and keep track of how a positively oriented frame returns to each $p_i$, i.e. with or without the orientation flipped (see \autoref{fig:two-possibilities}).
  \begin{figure}[h]
    \centering
    \incfig[0.5]{two-possibilities}
    \caption{Two possibilities for first return to $\eta_i$: on the left, the arc returns without the local orientation flipping, and on the right, the arc returns with the local orientation flipped.}
    \label{fig:two-possibilities}
  \end{figure}
  If the flow returns infinitely often without the orientation flipped, we join the endpoint $p_i$ to $p_0$ by going left along $\eta_i$ to get a simple closed curve that is two-sided.
  Furthermore, the geodesic tightening of the resulting curve is very close to the original curve, because the initial and final tangent vectors can be made arbitrarily close since they both face the ``up'' direction: the Anosov closing lemma then tells us that an orbit of the geodesic flow that approximately closes up can be perturbed by a small amount to exactly close up.
  This gives us a long geodesic that equidistributes with respect to the ergodic measure, and therefore an approximation by two-sided multicurves.

  If the flow does not return without the orientation flipped infinitely often, it must always return with the orientation flipped after some large enough $i_0$.
  In that case, consider the simple two-sided curves $c_i$ obtained by concatenating $a_i$ with the arc on $\eta_{i-1}$ joining $p_{i-1}$ and $p_i$ (see \autoref{fig:comes-with-flip}).
  \begin{figure}[h]
    \centering
    \incfig[0.8]{comes-with-flip}
    \caption{The curve $c_i$ is colored blue. Since the leaf from $p_0$ returns with the local orientation flipped to both $p_{i-1}$ and $p_{i}$, the curve $c_i$ is two-sided.}
    \label{fig:comes-with-flip}
  \end{figure}
  We have that as $i$ goes to $\infty$, the length of $c_i$ must go to $\infty$ as well, otherwise a subsequence would converge to a closed vertical curve starting at $p_0$, which cannot happen since the leaf through $p_0$ equidistributes.
  Also, note that the average of the curves $c_i$ weighted by their lengths for $i^{\prime} < i < i^{\prime \prime}$ where $i^{\prime \prime} \gg i^{\prime}$ is close to the ergodic measure, since we assumed that the leaf through $p_0$ equidistributes.
  This lets us invoke \autoref{lem:convexity-argument} to claim that there is a subsequence of $c_i$ whose orbit measures converge to the ergodic measure.
  Consequently, the geodesic representatives of $c_i$ converge to $\lambda_j$, since the geodesic tightening is close to the original curve, by the virtue of the initial and final tangent vectors being arbitrarily close.
  This resolves the two cases that can appear in the case of an orientable foliation, proving the result for case (ii).

  For case (iii), we define the points $p_i$, the nested intervals $\eta_i$, and the arcs $a_i$ in a similar manner as to case (ii).
  The key difference is that we no longer have that the foliation is orientable, which means the leaf can approach $p_i$ in one of four possible ways: from the ``up'' or the ``down'' direction, and with or without the orientation flipped.

  In case that the leaf approaches $p_i$ from the ``down'' direction without the orientation flipped infinitely often, the same closing argument as case (ii) works.
  Suppose now that the leaf approaches $p_i$ from the ``up'' direction, but without the orientation flipping, infinitely often.
  We then construct simple two-sided curves by concatenating the flow with the arc joining $p_i$ to $p_0$.
  % \begin{figure}[h]
  %   \centering
  %   \includegraphics{example-image-b}
  %   \caption{Constructing the curve $c_i$ when returning from the ``up'' direction.}
  %   \label{fig:coming-from-above}
  % \end{figure}
  While this curve does equidistribute with respect to the ergodic measure, it is not necessary that its geodesic tightening will do so.
  Denote the geodesic tightening by $c_i^{\prime \prime}$, and consider an intermediate closed curve $c_i^{\prime}$ in the same homotopy class as described in \autoref{lem:flat-length-tightening} and \autoref{lem:absolute-continuity}.
  We have by assumption that $\mu_{c_i}$ converges to an ergodic measure.
  If the ratio of the flat lengths of $c_i$ and $c_i^{\prime}$ is bounded away from $0$, possibly up to a subsequence, then \autoref{lem:absolute-continuity} tells us that $\mu_{c_i^{\prime}}$ converges to the ergodic measure, and \autoref{lem:flat-length-tightening} tells us the transverse measure associated to the hyperbolic geodesic representative is also ergodic.
  If $\lflat(c_i^{\prime \prime})$ is $o(\lflat(c_i))$, we still have that $\mu_{c_i^{\prime \prime}}$ converges up to a subsequence, since $l(c_i^{\prime \prime})$ goes to $\infty$, but it may converge to some other measure supported on the same underlying topological foliation.
  This proves the furthermore case of theorem.
  If $\lambda_j$ is actually uniquely ergodic, there is only one measure in the simplex of invariant probability measures, namely the uniquely ergodic one, and therefore $\mu_{g_i}$ is forced to converge to it.

  Suppose now that neither of the first two scenarios occur, i.e. the leaf returns to $p_i$ from the ``up'' or ``down'' direction, but with the orientation always flipped.
  We deal with this case like we did with the second subcase of case (ii).
  See \autoref{fig:case3-surgery} for the construction of the two-sided curves $c_i$.
  \begin{figure}[h]
    \centering
    \incfig[0.8]{case-3-surgery}
    \caption{Construction of the blue curve $c_i$ when the leaf always returns with orientation flipped from the ``up'' or ``down'' direction.}
    \label{fig:case3-surgery}
  \end{figure}
  We have that the geodesic tightenings of the curves $c_i$ are close to the original curve by the Anosov closing lemma, and that the weighted averages of the $c_i$ converge the ergodic measure, which means by \autoref{lem:convexity-argument} we have a subsequence $\mu_{c_i}$ that converges to the ergodic measure.
  This proves the result for case (iii), and therefore the theorem.
\end{proof}

% \paragraph{Idea of proof}
% First of all, note that one can approximate each minimal component on the subsurface on which the minimal components live in, and combine the rational approximations.
% This works because the rational approximations are also on difference subsurfaces, and thus have zero intersection number with each other.
% Case (i) is easy to deal with: since $\lambda_j \in \pmf^+(\no_g)$, the core curve is two-sided.
% Approximating that two-sided curve with the same two-sided curve, but with rational weights does the job.
% Cases (ii) and (iii) require more work.
% In case (ii), we first approximate the ergodic measures supported on the topological foliation $\lambda_j$.
% We do so by following a long orbit that equidistributes with respect to the ergodic measure, and closing the orbit up so that the resulting curve is simple and two-sided.
% To close up the orbit so that it still equidistributes with the same ergodic measure, we need to ensure that the foliation is orientable.
% Once we have rational approximations for all the ergodic measures supported on the topological foliation $\lambda_j$, we can obtain rational approximations for any convex combinations by taking large enough Dehn twists about each of the rational approximants.
% That proves the result for case (ii).
% For case (iii), we do the same thing as in case (ii), i.e. close up a long orbit so that it is simple and two-sided.
% Unique ergodicity ensures that the resulting curve equidistributes with respect to the uniquely ergodic measure.

\section{Upper bound for the limit set}
\label{sec:upper-bound-limit-set}

In this section, we prove that $\geolim(\mcg(\no_g))$ is contained in $\pmf^+(\no_g)$.
We do so by defining an $\mcg(\no_g)$-invariant subset $\systole(\no_g)$, and showing that the intersection of its closure with $\pmf(\no_g)$ is contained in $\pmf^+(\no_g)$.

\begin{definition}[One-sided systole superlevel set]
  For any $\varepsilon > 0$, the set $\systole(\no_g)$ is the set of all points in $\teich(\no_g)$ where the length of the shortest one-sided curve is greater than or equal to $\varepsilon$.
\end{definition}

We can state the main theorem of this section.
\begin{theorem}
  \label{thm:systole-closure}
  For any $\varepsilon > 0$, $\overline{\systole(\no_g)} \cap \pmf(\no_g)$ is contained in $\pmf^+(\no_g)$.
\end{theorem}
The key idea of the proof is proving a quantitative estimate on the Fenchel-Nielsen coordinates of points converging to points in $\pmf^-(\no_g)$.
\begin{proposition}
  \label{prop:pinching}
  Let $\{m_i\}$ be a sequence of points in $\teich(\no_g)$ converging to a projective measured foliation $[\lambda]$.
  If $p$ is a one-sided atom of $\lambda$, for any Fenchel-Nielsen coordinate chart containing $p$ as a cuff, the length coordinate of $p$ goes to $0$.
\end{proposition}
\begin{proof}
  Consider the following decomposition of the measured foliation $\lambda$.
  \begin{align*}
    \lambda = 1 \cdot p + \lambda_{\mathrm{at}} + \lambda_{\mathrm{Leb}}
  \end{align*}
  Here, $\lambda_{\mathrm{at}}$ are the minimal components on periodic components other than $p$, i.e. cylinders and Möbius strips, and $\lambda_{\mathrm{Leb}}$ are non-periodic minimal components.
  In the above expression, $p$ is the one-sided curve considered as a measured foliation (since we're picking a representative of $[\lambda]$, we can pick one such that $p$ has weight $1$)

  Pick simple closed curves $p_0$, $p_1$, and $p_2$, where $p_0$ is the curve $p$, and $\{p_0, p_1, p_2\}$ bound a pair of pants.
  Furthermore, we impose the following conditions on $p_1$ and $p_2$.
  \begin{align*}
    i(p_1, \lambda_{\mathrm{at}}) &= 0 \\
    i(p_2, \lambda_{\mathrm{at}}) &= 0
  \end{align*}
  Note that this can always be done, by deleting the support of $\lambda_{\mathrm{at}}$, and looking at the resulting subsurfaces.
  Note that neither $p_1$ nor $p_2$ can be the same as $p_0$, since $p_0$ is one-sided.
  We focus our attention to this pair of pants, and consider the curves on it, labelled in \autoref{fig:pants}.
  \begin{figure}[h]
    \centering
    \incfig[0.8]{pants}
    \caption{The curves restricted to a pair of pants.}
    \label{fig:pants}
  \end{figure}

  The curve labelled $p_3$ has intersection number at least $1$ with $\lambda$ which means its length should be going to $\infty$.
  We can bound the length of $p_3$ above and below via the lengths of the orthogeodesic $p_4$ and the length of $p_0$.
 \begin{equation}
   \label{eq:up-1}
   \lhyp(p_0) \leq \lhyp(p_3) \leq \lhyp(p_4) + \lhyp(p_0)
 \end{equation}
 Observe that we get the upper bound by observing that the red and cyan arcs are isotopic to $p_3$ relative to their endpoints being fixed.
 The cyan arcs have length at most $\lhyp(p_0)$, in this setting; if one allowed a twist parameter, the length of the cyan arcs would be proportional to the twist parameters.
 The point of this inequality is that we can estimate $\lhyp(p_4)$ using $\lhyp(p_0)$, $\lhyp(p_1)$ and $\lhyp(p_2)$ via hyperbolic trigonometry.
 Cut the pair of pants along the seams, to get a hyperbolic right-angled hexagon, pictured in \autoref{fig:hexagon}.
 \begin{figure}[h]
   \centering
   \incfig[1]{hexagon}
   \caption{The right angled hexagon obtained by cutting the pants along the seams.}
   \label{fig:hexagon}
 \end{figure}

 To get good estimates on $\lhyp(p_4)$, we need a universal lower bound on the fraction $f$ as we move in the Teichm\"uller space.
 The analysis splits up into two cases, but it's not \emph{a~priori} clear that these two cases are exhaustive.
 We will deal with the two cases, and then show that any other case can be reduced to the second case by changing $p_1$ and $p_2$.

\subsection*{Case I}
\label{case1}
We're in this case if $p_1$ and $p_2$ don't intersect the foliation $\lambda$ at all.
\begin{align*}
  i(p_1, \lambda) &= 0 \\
  i(p_2, \lambda) &= 0
\end{align*}
In this case, we can pass to a subsequence of $\{m_i\}$ such that the corresponding values of $f$ are always greater than $\frac{1}{2}$ or less than $\frac{1}{2}$.
In the former case, we focus on $p_1$, and in the latter case, we focus on $p_2$.
Without loss of generality, we'll suppose $f \geq \frac{1}{2}$.
In that case, we cut along the orthogeodesic $p_4$, and get a hyperbolic right-angled pentagon, which is the top half of \autoref{fig:hexagon}.
We can relate the lengths of $\lhyp(p_0)$, $\lhyp(p_1)$, and $\lhyp(p_4)$ using the following identity for hyperbolic right-angled pentagons (see \cite{thurston1979geometry} for the details).
\begin{align}
  \label{eq:pentagon}
  \sinh\left( f \cdot \lhyp(p_0) \right)  \cdot \sinh\left( \frac{\lhyp(p_4)}{2} \right)  = \cosh\left( \frac{\lhyp(p_1)}{2} \right)
\end{align}
Now suppose that $\lhyp(p_0)$ does not go to $0$.
Then we must have that for all $m_i$, $\lhyp(p_0) \geq 2\varepsilon$ for some $\varepsilon > 0$. By the lower bound on $f$, we have that the first term on the left hand side of the above expression is bounded below by $\varepsilon$.
Rearranging the terms gives us the following upper bound on $\lhyp(p_4)$.
\begin{equation}
  \label{eq:up-2}
  \lhyp(p_4) \leq 2 \cdot \sinh^{-1} \left( \frac{\cosh \left( \frac{\lhyp(p_1)}{2} \right)}{\varepsilon} \right)
\end{equation}
Using \eqref{eq:up-1} and \eqref{eq:up-2}, we get an upper bound for $\lhyp(p_3)$.
\begin{equation}
  \label{eq:up-3}
  \lhyp(p_3) \leq \lhyp(p_0) +
  2 \sinh^{-1} \left( \frac{\cosh \left( \frac{\lhyp(p_1)}{2} \right)}{\varepsilon} \right)
\end{equation}

Since $\frac{i(p_0, \lambda)}{i(p_3, \lambda)} = 0$, as $\{m_i\}$ approaches $\lambda$, the ratio of lengths of $p_0$ and $p_3$ approach $0$.
\begin{equation}
  \label{eq:up-4}
  \lim_{i \to \infty} \frac{\lhyp(p_0)}{\lhyp(p_3)} = 0
\end{equation}
Using \eqref{eq:up-3}, we can find a lower bound for $\frac{\lhyp(p_0)}{\lhyp(p_3)}$, which goes to $0$ by \eqref{eq:up-4}.
\begin{equation}
  \label{eq:up-5}
  \lim_{i \to \infty} \frac{\lhyp(p_0)}{  \lhyp(p_0) +
  2 \sinh^{-1} \left( \frac{\cosh \left( \frac{\lhyp(p_1)}{2} \right)}{\varepsilon} \right)} = 0
\end{equation}
Since we assumed that the length coordinate $\lhyp(p_0)$ is bounded away from $0$, the only way the above expression can go to $0$ if $\lhyp(p_1)$ goes to $\infty$.
This is where the hypotheses of the Case I come in.
Since $i(p_1, \lambda)$ is $0$, the following equality must hold.
\begin{equation}
  \label{eq:up-6}
  \lim_{i \to \infty} \frac{\lhyp(p_1)}{\lhyp(p_3)} = 0
\end{equation}
This means the lower bound for $\frac{\lhyp(p_1)}{\lhyp(p_3)}$ must go to $0$.
\begin{equation}
  \label{eq:up-7}
   \lim_{i \to \infty} \frac{\lhyp(p_1)}{  \lhyp(p_0) +
  2 \sinh^{-1} \left( \frac{\cosh \left( \frac{\lhyp(p_1)}{2} \right)}{\varepsilon} \right)
  } = 0
\end{equation}
But this can't happen if $\lhyp(p_1)$ approaches $\pm \infty$.
This contradiction means our assumption that both $\lhyp(p_0)$ and $\tau(p_0)$ were bounded away from $0$ and $\pm \infty$ must be wrong, and thus proves the result in Case I.

\subsection*{Case II}
We're in this case if the following inequality holds.
\begin{equation}
  \label{eq:up-8}
  0 < i(p_1, \lambda) < 1
\end{equation}
The picture in this case looks similar to \autoref{fig:hexagon}.
However, we can't necessarily pass to a subsequence where $f \geq \frac{1}{2}$ (and the trick of working with $1-f$ won't work, since we know nothing about $p_2$).
This is one of the points where the hypothesis on $p_1$ comes in.
Since $\frac{i(p_2, \lambda)}{i(p_1, \lambda)}$ is finite, we must have that the ratio of lengths $\frac{\lhyp(p_2)}{\lhyp(p_1)}$ approaches some finite value as well.
The fraction $f$ is a continuous function of $\frac{\lhyp(p_2)}{\lhyp(p_1)}$, approaching $0$ only as the ratio approaches $\infty$ (this follows from the same identity as \eqref{eq:pentagon}). Since the ratio approaches a finite value, we have a positive lower bound $f_0$ for $f$.

Assuming as before that $\lhyp(p_0)$ is bounded away from $0$, and $\tau(p_0)$ bounded away from $\pm \infty$, and repeating the calculations of the previous case, we get the following two inequalities.
\begin{equation}
  \label{eq:up-9}
  \frac{\lhyp(p_1)}{\lhyp(p_3)} \geq \frac{\lhyp(p_1) }{\lhyp(p_0) +
  2 \sinh^{-1} \left( \frac{\cosh\left( \frac{\lhyp(p_1)}{2} \right)}{f_0 \varepsilon} \right)
  }
\end{equation}
\begin{equation}
  \label{eq:up-10}
  \frac{\lhyp(p_0)}{\lhyp(p_3)} \geq \frac{\lhyp(p_0)}{\lhyp(p_0) +
  2 \sinh^{-1} \left( \frac{\cosh\left( \frac{\lhyp(p_1)}{2} \right)}{f_0 \varepsilon} \right)
  }
\end{equation}
The right hand side of \eqref{eq:up-10} must approach $0$, and that forces either $\lhyp(p_1)$ or $\lhyp(p_2)$ to approach $\infty$.
But that means the right hand term of \eqref{eq:up-9} must approach $1$, which cannot happen, by the hypothesis of case II.
This means $\lhyp(p_0)$ goes to $0$, proving the result in case II.

\subsection*{Reducing to case II} Suppose now that both $p_1$ and $p_2$ have an intersection number larger than $1$ with $\lambda$.
We can modify one of them to have a small intersection number with $\lambda$.
First, we assume that $\lambda_{\mathrm{Leb}}$ is supported on a single minimal component, i.e. every leaf of $\lambda_{\mathrm{Leb}}$ is dense in the support.
We now perform a local surgery on $p_1$: starting at a point on $p_1$ not contained in the support of $\lambda_{\mathrm{Leb}}$, we follow along until we intersect $\lambda_{\mathrm{Leb}}$ for the first time.
We denote this point by $\alpha$.
We now go along $p_1$ in the opposite direction, until we hit the support of $\lambda_{\mathrm{Leb}}$ again, but rather than stopping, we keep going until the arc has intersection number $0 < \delta < 1$ with $\lambda_{\mathrm{Leb}}$.
We then go back to $\alpha$, and follow along a leaf of $\lambda_{\mathrm{Leb}}$ rather than $p_1$, until we hit the arc.
This is guaranteed to happen by the minimality of $\lambda_{\mathrm{Leb}}$.
Once we hit the arc, we continue along the arc, and close up the curve.
This gives a new simple closed curve which intersection number with $\lambda$ is at most $\delta$.
This curve is our replacement for $p_1$ (see \autoref{fig:first-return-map} for a schematic of this local surgery).
If $\lambda_{\mathrm{Leb}}$ is not minimal, we repeat this process for each minimal component. We pick $p_2$ in a manner such that $p_0$, $p_1$, and $p_2$ bound a pair of pants.
Since $\delta < 1$, we have reduced to case II.
This concludes the proof of the theorem.
\begin{figure}[h]
  \centering
  \incfig[0.5]{first-return-map}
  \caption{Using the first return map on $\lambda$ to modify $p_1$.}
  \label{fig:first-return-map}
\end{figure}
\end{proof}

\begin{remark}[On the orientable version of \autoref{prop:pinching}]
  The same idea also works in the orientable setting, although the analysis of the various cases gets a little more delicate.
  The first change one needs to make is in the statement of the proposition: we no longer need to require $p$ to be a one-sided atom, and correspondingly, either the length coordinate $\lhyp(p)$ can go to $0$, or the twist coordinate $\tau(p_0)$ can go to $\pm \infty$.
  To see how the twist coordinate enters the picture, observe that \eqref{eq:up-1}, which was the main inequality of the proof, turns into the following in the orientable version.
  \begin{equation}
    \label{eq:up-11}
    \lhyp(p_4)  \leq \lhyp(p_3) \leq \tau(p_0) + \lhyp(p_4)
  \end{equation}
  Here, $\tau(p_0)$ is the twist parameter about $p_0$, and $p_4$ is the orthogeodesic multi-arc (there may be one or two orthogeodesics, depending on the two cases described below).

  The proof splits up into two cases, depending on whether both sides of $p$ are the same pair of pants, or distinct pairs of pants.
  This was not an issue in the non-orientable setting, since $p$ was one-sided.
  If both sides of $p$ are the same pair of pants, then the analysis is similar to what we just did, since the curve $p_3$ stays within a single pair of pants.
  In the other, $p_3$ goes through two pair of pants, and its length is a function of the twist parameters, as well the cuff lengths of four curves, rather than two curves, the four curves being the two remaining cuffs of each pair of pants.
  The analysis again splits up into two cases, depending on the intersection number of the cuffs with $\lambda$, but reducing all the other cases to case II becomes tricky because we need to simultaneously reduce the intersection number of two curves, rather than one, as in the non-orientable setting.
  This added complication obscures the main idea of the proof, which is why we chose to only prove the non-orientable version.
\end{remark}

This quantitative estimate of \autoref{prop:pinching} gives us an easy proof of \autoref{thm:systole-closure}.
\begin{proof}[Proof of \autoref{thm:systole-closure}]
  Suppose that the theorem were false, and there was a foliation $[\lambda] \in \pmf^-(\no_g)$ in the closure of $\systole(\no_g)$.
  Suppose $p$ is a one-sided atom in $\lambda$.
  Then \autoref{prop:pinching} tells us that the hyperbolic length of $p$ goes to $0$, but the length of $p$ must be greater than $\varepsilon$ in $\systole(\no_g)$.
  This contradicts our initial assumption, and the closure of $\systole(\no_g)$ can only intersect $\pmf(\no_g)$ in the complement of $\pmf^-(\no_g)$.
\end{proof}

\begin{corollary}
  \label{cor:geolimset}
  The geometric limit set $\geolim(\mcg(\no_g))$ is contained in $\pmf^+(\no_g)$.
\end{corollary}
\begin{proof}
  Every point $p \in \teich(\no_g)$ is contained in $\systole(\no_g)$ for some small enough $\varepsilon$.
  This means $\Lambda_{\mathrm{geo}, p}(\mcg(\no_g))$ is contained in $\pmf^+(\no_g)$ by \autoref{thm:systole-closure}.
  Taking the union over all $p$ proves the result.
\end{proof}

\section{Domain of discontinuity}
\label{sec:domain-discontinuity}

In this section, we prove that the domain of discontinuity for the action of $\mcg(\no_g)$ on $\pmf(\no_g)$ is empty.
This follows fairly easily from \autoref{conj:gendulphe-1}, but because we only have a slightly weaker version of the result, i.e. \autoref{thm:rational-approximation}, we need to work a little harder.
However, the techniques used in the proof of \autoref{thm:rational-approximation} are robust enough to prove the result in this section.
\begin{theorem}
  \label{thm:dod-is-empty}
  The domain of discontinuity for the $\mcg(\no_g)$ action on $\pmf(\no_g)$ is empty.
\end{theorem}

We prove this result by showing that for any measured foliation $\lambda$, there exists a mapping class $\varphi$ such that $\varphi$ moves $\lambda$ by an arbitrarily small amount.
\begin{proposition}
  \label{prop:mapping-class-estimate}
  Let $\lambda$ be a measured foliation.
  Then for any simple closed curve $\kappa$ and any $\delta > 0$, there exists a $\varphi \in \mcg(\no_g)$ such that the following inequalities hold.
  \begin{align*}
    i(\lambda , \kappa) - \delta
    \leq i(\varphi(\lambda), \kappa)
    \leq i(\lambda , \kappa) + \delta
  \end{align*}
\end{proposition}
To prove this proposition, we will need to estimate how Dehn twists change intersection numbers.
\begin{lemma}
  \label{lem:dehn-twist-estimate}
  Let $\lambda$ be any measured foliation, $\gamma$ any two-sided simple curve, and $\kappa$ any simple curve (not necessarily two-sided).
  Denote the Dehn twist about $\gamma$ by $T_{\gamma}$; the intersection number of $T_\gamma(\lambda)$ and $\kappa$ is bounded by the following intersection numbers.
  \begin{align*}
    i(\lambda, \kappa) - i(\lambda, \gamma) \cdot i(\gamma, \kappa)
    \leq i(T_{\gamma}, \kappa)
    \leq i(\lambda, \kappa) + i(\lambda, \gamma) \cdot i(\gamma, \kappa)
  \end{align*}
\end{lemma}
\begin{proof}
  We approximate $\lambda$ by simple curves, possibly one-sided, and use Proposition 3.4 of \cite{farb2011primer}, which is the same inequality, but with $\lambda$ replaced with a simple curve.
\end{proof}

\begin{proof}[Proof of \autoref{prop:mapping-class-estimate}]
  Consider a projective measured lamination $[\lambda] \in \pmf(\no_g)$.
  If $[\lambda]$ contains one-sided leaves, we delete those one-sided curves and work with the resulting subsurface.
  If the restriction of $\lambda$ to the resulting subsurface is empty, then any mapping class on the subsurface does not change $\lambda$, and the result holds.
  If the restriction of $\lambda$ to the subsurface is non-empty, and $\lambda$ can be approximated by simple two-sided curves and \autoref{thm:equivalence-of-limit-sets} tells us that it can also be approximated by stable foliations of pseudo-Anosov maps on the subsurface.
  For any small neighbourhood of $[\lambda]$, pick a pseudo-Anosov $\varphi$ whose stable foliation lies in that neighbourhood.
  Then the north-south dynamics of pseudo-Anosov maps tells us that $\varphi([\lambda])$ will be in the same neighbourhood, proving the result in this case.

  Suppose now that the restriction of the foliation is not approximable by simple two-sided curves.
  Then there exists some minimal component that is not approximable by two-sided curves.
  \autoref{lem:convex-combinations} tells us that there is some ergodic measure supported on the underlying topological foliation that is not approximable by two-sided curves either.
  Let $p$ be a generic point for this measure, and let $\eta$ be a transverse arc whose left endpoint is $p$.
  Pick a flat structure on this surface such that $\lambda$ is the vertical foliation, and $\eta$ is a horizontal arc of length $1$.
  The only way the ergodic measure is not approximable by a two-sided curve is the situation described in the proof of \autoref{thm:rational-approximation} where a leaf leaving $p$ from the top returns infinitely often from the top again, with the orientation remaining unflipped.
  Recall that in this situation, we have an arc $a$ a leaving $p$ from the top, following the foliation, terminating at a point $q$ that is $\varepsilon$ distance to the right of $p$, and it comes back from the top with the orientation unflipped.
  We concatenate this arc with the segment along $\eta^{\prime}$ joining $p$ and $q$.
  This is a simple two-sided curve, which we will call $c$ (this curve $c$ depends on $\varepsilon$, but we will suppress the dependence in the notation).
  Let $c^{\prime}$ be the tightening of this curve described in the paragraph before \autoref{lem:flat-length-tightening}.
  If we do not have that $\lflat(c^{\prime})$ is $o(\lflat(c))$, then \autoref{lem:absolute-continuity} tells us that the ergodic measure can be approximated by simple two-sided curves.
  That means $\varepsilon \cdot \lflat(c^{\prime})$ goes to $0$ as $\varepsilon$ goes to $0$.

  We now relate $\lflat(c^{\prime})$ with $i(c^{\prime}, \kappa)$ via the following inequality, where $c^{\prime}$ can be any closed curve, and $j_{\kappa}$ is a constant depending on $\kappa$ and the flat geometry of the surface.
  \begin{align}
    \label{eq:13}
    i(c^{\prime}, \kappa) \leq j_{\kappa} \cdot \lflat(c^{\prime})
  \end{align}
  The constant $j_{\kappa}$ can be computed by cutting the surface along $\kappa$, and computing the length of the shortest essential arc starting and ending at the boundary component $\kappa$.

  We pick an $\varepsilon$ and curve $c$ such that $\varepsilon \cdot \lflat(c^{\prime}) < \frac{\delta}{j_{\kappa}}$.
  We Dehn twist about the curve $c$: \autoref{lem:dehn-twist-estimate} tells us that the error term is $i(\lambda, c) \cdot i(c, \kappa) = \varepsilon \cdot i(c, \kappa)$.
  We bound the intersection number by $j_{\kappa} \cdot \lflat(c^{\prime})$, which bounds the error term by $\delta$, proving the result in the case when $\lambda$ is not approximable by two-sided curves.

  This handles all the cases, and proves the proposition.
\end{proof}

\begin{proof}[Proof of \autoref{thm:dod-is-empty}]
  Pick any small open set $U$ in $\pmf(\no_g)$, and any arbitrary local metric such that $U$ has diameter at least $2$.
  Pick a sequence of mapping classes $\{\varphi_i\}$ that move a point by distance at most $\frac{1}{2^i}$, using \autoref{prop:mapping-class-estimate}.
  Then for all $i$, $\varphi_i(U) \cap U$ is non-empty.
  This proves the result.
\end{proof}

\section{Failure of quasi-convexity for $\systole$}
\label{sec:fail-quasi-conv}

In the setting of Teichm\"uller geometry, convexity is usually too strong of a requirement.
For instance, metric balls in Teichm\"uller space are not convex, but merely quasi-convex (see \autocite{lenzhen2011length}).
\begin{definition}[Quasi-convexity]
  A subset $S$ of $\overline{\teich(S)}$ is said to be quasi-convex if there is some uniform constant $R > 0$ such that the geodesic segment joining any pair of points in $S$ stays within distance $R$ of $S$.
\end{definition}

Our goal for this section will be to prove the following theorem.
\begin{theorem}
  \label{thm:qc-fail}
  The set $\systole(\no_g)$ is not quasi-convex for any $\varepsilon > 0$.
\end{theorem}
The proof proceeds by constructing surfaces in the weak convex hull of $\pmf^+(\no_g)$ that have arbitrarily small one-sided curves.
We use this fact, along with estimates relating Teichm\"uller distances and hyperbolic lengths to obtain failure of quasi convexity.

We begin by finding short curves in the weak convex hull of $\pmf^+(\no_g)$, for $g \geq 4$\footnote{For $g = 3$, the weak convex hull is empty, since no two two-sided foliations can fill $\no_3$.}.
\begin{proposition}
  \label{prop:very-short-curves}
  For all $g \geq 4$ and any $\varepsilon > 0$, there exist a point $p$ in the weak convex hull of $\pmf^+(\no_g)$ such that some one-sided curve has length less than $\varepsilon$ with respect to the metric on $p$.
\end{proposition}

To prove this result, we will need a lemma that relates hyperbolic and flat lengths.
\begin{lemma}
  \label{lem:schwarz}
  Let $q$ be an area $1$ DQD on $\no_g$, and consider the unique hyperbolic metric with the same conformal structure.
  Let $A$ be a primitive annulus in $q$, i.e. an annulus whose interior does not pass  through a singularity of the flat metric.
  Let the modulus of $A$ be $m$.
  Then the hyperbolic length of the isotopy class of the core curve of the annulus is at most $\frac{\pi}{m}$.
\end{lemma}
\begin{proof}[Sketch of proof]
  Without loss of generality, we can pass to the orientable double cover.
  This changes the hyperbolic lengths by at most a factor of two.
  Consider the interior of the annulus as a Riemann surface, and put the unique hyperbolic metric on that surface.
  With respect to this hyperbolic metric, the length of the core curve is $\frac{\pi}{m}$. Since the interior doesn't contain any singularities, the inclusion map is holomorphic, and holomorphic maps are distance reducing with respect to the hyperbolic metric.
  This proves the result.
\end{proof}
To exhibit short curves in the weak convex hull of $\pmf^+(\no_g)$, we will find annuli of large modulus whose core curves are the curves we're looking for.
Lemma \ref{lem:schwarz} will then gives us that the hyperbolic length of those curves are short.

\begin{proof}[Proof of \autoref{prop:very-short-curves}]
  We will construct DQDs on $\no_g$ such that one can find annuli of arbitrarily high modulus whose core curves are isotopic to a one-sided closed curve traversed twice.
  By \autoref{lem:schwarz}, the core curves, and therefore the associated one-sided curves will have arbitrarily small hyperbolic lengths.
  We will also show that the horizontal and vertical foliations of the flat surfaces we
  construct are in $\overline{\pmf^+(\no_g, \QQ)}$.
  That will then conclude the proof.

  We construct the examples for $g=4$ and $g=5$ by hand, and then induct from there on.

\subsection*{The $\mathbf{g=4}$ case}
Consider the DQD on $\no_4$ pictured in \autoref{fig:no4-dqd}.
Note that the edges labelled $b$, $d$, $b^{\prime}$ and $d^{\prime}$ are identified via the map $z \mapsto -\overline{z} + c$.
If the lengths of $b$ and $b^{\prime}$, $c$ and $c^{\prime}$, and $d$ and $d^{\prime}$ are set to be equal, that will ensure that the vertical foliation is in $\pmf^+(\no_g, \QQ)$.
The horizontal foliation is also in $\pmf^+(\no_g, \QQ)$ since it is periodic as well with the core curve being a two-sided curve.
This shows that the corresponding point in $\teich(\no_4)$ is indeed in the weak convex hull of $\overline{\pmf^+(\no_4, \QQ)}$.
Now observe the annulus bounded by the two circles in the picture.
The core curve $\gamma$ of that annulus is isotopic to the curve obtained by traversing $b^{\prime}$ twice.
That means if the hyperbolic length of $\gamma$ is $l$, then the hyperbolic length of $b^{\prime}$ is $\frac{l}{2}$.
Also, by making the length of $b$ and $b^{\prime}$ smaller, and correspondingly making the inner radius of the annulus smaller, we can make the modulus are large as we like, which still stays in the weak convex hull of $\overline{\pmf^+(\no_4, \QQ)}$.
This proves the result for the $g=4$ case.
\begin{figure}[h]
  \centering
  \incfig[0.7]{no4-dqd}
  \caption{A DQD on $\no_4$ with an annulus.}
  \label{fig:no4-dqd}
\end{figure}

\subsection*{The $\mathbf{g=5}$ case}
Consider the DQD on $\no_5$ pictured in \autoref{fig:no5-dqd}.
We require that the edges labelled $a$, $b$, $c$, and $d$ are glued via the map $z \mapsto z + c$.
Furthermore, we require that the lengths of $a$, $b$, $c$ and $d$ are equal, and that every vertical geodesic going through the bottom slit also goes through the top slit, which is glued to itself via the map $z \mapsto -\overline{z} + c$.
Finally, we also require that the top and bottom edges are completely horizontal, and the left and right edges are vertical.

Consider the first return map to the lower slit with respect to the vertical flow.
Every saddle connection gets mapped to another saddle connection, with its direction reversed, since they all went through the top slit.
The permutation of the saddle connections is the following.
\begin{align*}
  a &\mapsto -d \\
  d &\mapsto -c \\
  c &\mapsto -b \\
  b &\mapsto -a
\end{align*}
Note that this is an order $4$ permutation.
What that means is that if there were a closed vertical geodesic passing through the one-sided curve, it would pass through the one-sided curve exactly $4$ times.
Since that's even, this would mean the closed geodesic was two-sided.
This shows that the vertical foliation is completely periodic and contains no one-sided leaf. It's easy to see that the horizontal foliation is periodic without any one-sided leaves as well.
That means the corresponding point in $\teich(\no_5)$ lies in the weak convex hull of
$\overline{\pmf^+(\no_5, \QQ)}$.
Observe now the annulus in the figure.
The core curve is the double of the one-sided curve $e$.
Furthermore, if we make both the slits short, while maintaining all the other properties we listed earlier, we can make the inner radius of the annulus small, getting an annulus with arbitrarily large modulus.
This proves the result for $g=5$.
\begin{figure}[h]
  \centering
  \incfig[1.4]{no5-dqd}
  \caption{A DQD on $\no_5$ with an annulus.}
  \label{fig:no5-dqd}
\end{figure}

\subsection*{Induction step}
To prove the result for any higher even genus $g$, cut $\frac{g-4}{2}$ vertical slits on the flat surface pictured in \autoref{fig:no4-dqd}, and glue in a torus.
This will ensure the resulting surface has the right genus, without changing the upper bound on the length of the one-sided curve that's getting short.
Furthermore, this only adds other minimal components to the vertical flow, all of which are coming from orientable subsurfaces, and therefore in $\overline{\pmf^+(\no_g, \QQ)}$.
The horizontal foliation also stays in $\overline{\pmf^+}(\no_g)$ since the horizontal foliation lives on an orientable subsurface.
Performing the same construction to the example for $\no_5$ lets us achieve any odd genus, and proves the proposition.
\end{proof}

\begin{remark}
  In the process of proving the proposition, we constructed a sequence on points in $\teich(\no_g)$ in which a one-sided curve gets progressively shorter.
  In fact, this also shows that this sequence of points converges to the projective class of the one-sided geodesic that was getting short.
  This is quite unexpected, because one would normally expect the closure of the weak convex hull of a closed set to be the set itself.
  What we have here is that the closure of the weak convex hull of $\pmf^+(\no_g)$ intersected with the boundary contains points other than $\pmf^+(\no_g)$.
\end{remark}

To relate Teichm\"uller distance to hyperbolic lengths, we need Wolpert's lemma (\autocite{wolpert1979length})
\begin{lemma}[Wolpert's Lemma]
  Let $M$ and $M^{\prime}$ be two points in $\teich(\os_g)$, and let $\gamma$ be a simple closed curve on $\os_g$.
  Let $R$ be the Teichm\"uller distance between $M$ and $M^{\prime}$. Then the ratio of the hyperbolic length
  of $\gamma$ and $R$ are related by the following inequalities.
  \begin{align*}
    \exp(-2R) \leq \frac{\ell_{\mathrm{hyp}}(M, \gamma)}{\ell_{\mathrm{hyp}}(M^{\prime}, \gamma)} \leq \exp(2R)
  \end{align*}
\end{lemma}

Using Proposition \ref{prop:very-short-curves} and Wolpert's lemma, we can prove Theorem
\ref{thm:qc-fail}.
\begin{proof}[Proof of Theorem \ref{thm:qc-fail}]
  Suppose that $\systole(\no_g)$ was indeed quasi-convex.
  That would mean that there exists some $R > 0$,
  depending on $\varepsilon$ such that everything in the weak convex hull of $\pmf^+(\no_g)$ was within $R$
  distance of some point in $\systole(\no_g)$.  Proposition \ref{prop:very-short-curves} lets us construct a
  sequence $\{M_i\}$ in the weak convex hull such that for a fixed one-sided curve $\gamma$, the lengths
  $\ell_{\mathrm{hyp}}(M_i, \gamma)$ go to $0$. On the other hand, using quasi-convexity, we have a sequence
  $\{M_{i}^{\prime}\}$ in $\systole(\no_g)$ such that $d_{\mathrm{Teich}}(M_i, M_i^{\prime})$ is less than
  $R$. Furthermore, since $M_i^{\prime} \in \systole(\no_g)$, $\ell_{\mathrm{hyp}}(M_i^{\prime}, \gamma)$ is at
  least $\varepsilon$.

  To use Wolpert's lemma, we need to consider the image of $\systole(\no_g)$ and the weak convex hull of
  $\pmf^+(\no_g)$ in $\teich(\os_{g-1})$. This is an isometric embedding, so the distance between these two
  sets will be bounded above by $R$. Passing to the double cover, we will need to work with the lift
  $\wt{\gamma}$ of $\gamma$, which will have double the hyperbolic length.  Since the length of $\gamma$ goes
  to $0$ in the sequence $\{M_i\}$, the length of $\wt{\gamma}$ in the image $\wt{M_i}$ of the sequence will
  also go to $0$. Similarly, since the length of $\gamma$ is bounded below by $\varepsilon$ in
  $\{M_{i}^{\prime}\}$, the length of $\wt{\gamma}$ will be bounded below by $2 \varepsilon$.

  Using Wolpert's lemma, we have the following inequality.
  \begin{equation}
    \label{eq:20}
    \frac{\ell_{\mathrm{hyp}}(M_i^{\prime})}{\ell_{\mathrm{hyp}}(M_i)} \leq \exp(2R)
  \end{equation}
  The numerator of the left hand side is bounded below by $\varepsilon$, while the denominator goes to $0$, but that
  can't happen since the upper bound is finite, and independent of $i$.
  This contradiction shows that $\systole(\no_g)$ cannot be quasi-convex.
\end{proof}

\printbibliography

\end{document}
