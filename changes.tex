\documentclass[12pt, reqno]{amsart}
\pdfoutput=1
\usepackage{mathpreamble}

\title{Changes to ``The limit set of non-orientable mapping class groups''}
\author{Sayantan Khan}
\address{Department of Mathematics, University of Michigan, Ann Arbor, MI}
\email{\href{mailto:saykhan@umich.edu}{saykhan@umich.edu}}
\urladdr{\url{https://www-personal.umich.edu/~saykhan/}}


\date{\today}

\begin{document}

\maketitle

\section{Corresponding results about $\mcg(\os_g)$}
\label{sec:corr-results-about}

\textbf{Referee comment:} It would be helpful if the author can include corresponding results about $\mcg(\os_g)$ in the 2\textsuperscript{nd} chapter as comparison.

\textbf{Edits/Response:} Added Theorem 2.6 and Corollary 2.8 to illustrate what happens in the case of orientable surfaces.

\section{Different notions of limit set}
\label{sec:diff-noti-limit}

\textbf{Referee comment:} You have two different notions of limit set, could you state more precisely what did Erlandsson, Gendulphe, Pasquinelli, and Suoto prove? According what you wrote, they prove $\Gamma_{\text{dyn}} = \Gamma_{\text{geom}} = \pmf^+(\no_g)$, right? Is it possible to improve your method to conclude the same?

\textbf{Edits/Response:} Added a sentence and paragraph in the corresponding section about what the four author paper proves, as well as why the methods in this paper cannot quite prove the full result.

\section{Definition of $p_3$ in Section 4}
\label{sec:defin-p_3-sect}

\textbf{Referee comment:} What is the definition of $p_3$ in Chapter 4? Page 16, last paragraph, which means its length should be going to $\infty$, why? Your formula is a bit weird, the left side is $p_4$?

\textbf{Edits/Response:} The curve $p_3$ is now defined after the proof of Claim 4.4. The claim itself shows that $p_3$ is well-defined.
The claims 4.4 and 4.5 are required to prove inequality (11), which has been changed to be just an upper bound for $\ell(p_3)$, which is all we need. And $\ell(p_3)$ goes to $\infty$ because it intersects $p_0$, which is a component of the limiting foliation.

\section{Weird notation in chapter 3}
\label{sec:weird-notat-chapt}

\textbf{Referee comment:} The whole chapter 3 has weird notation with $\lhyp(p_i)$. It would make more sense to use $\ell_i$ to represent the hyperbolic length w.r.t the metric $m_i$.
Also please reorganize your computation to make it more logical. The final contradiction is also weird, what do you mean by this can't happen if $\lhyp(p_i)$ approaches $\infty$? I thought it should be zero by your assumption. Your equation (18) should be interpreted algebraically in a better way.

\textbf{Remark:} I think the referee meant weird notation in chapter 4, since $\lhyp$ shows up in that chapter.

\textbf{Edits/Response:} All the instances of $\lhyp(p_k)$ have been changed to $\ell_i(p_k)$, which denotes the length of the curve $p_k$ (where $k \in \left\{ 0, 1, 2, 3, 4 \right\}$) with respect to the metric $m_i$.
Also, two additional inequalities have been added that clarify the estimates in the section. The estimate involves first showing that $\ell_i(p_1)$ goes to $\infty$, and the using that fact to show that $\frac{\ell_i(p_1)}{\ell_i(p_3)}$ cannot go to $0$, violating the assumptions made in case I.
Equation (17) was added to make equation (19) (which was previously (18)) easier to understand.

\section{Margulis lemma type result}
\label{sec:margulis-lemma-type}

\textbf{Referee comment:} I understand that you use hyperbolic geometry estimates to argue, but if you can, please have a more intuitive explanation. Is it similar to Margulis lemma type result?

\textbf{Edits/Response:}

\section{Lower genera construction}
\label{sec:lower-genera-constr}

\textbf{Referee comment:} Why do you not have construction for lower genera?

\section{Counting problem for $\no_g$}
\label{sec:counting-problem}

\textbf{Referee comment:} Could you state more precisely how does your result (or related result about limit set) help with the counting problem for $\no_g$?

\section{Link between Theorem 5.2 and limit set}
\label{sec:link-between-theorem}

\textbf{Referee comment:} In the introduction, you said on page 2, that the key idea involved in Theorem 5.2 is to determine the limit set; where do you use this in the proof? If not, can you find a different bridge between these two results.



\end{document}