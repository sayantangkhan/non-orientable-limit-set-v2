\documentclass[12pt, reqno]{amsart}
\pdfoutput=1
\usepackage{mathpreamble}

\title{Changes to ``The limit set of non-orientable mapping class groups''}
\author{Sayantan Khan}
\address{Department of Mathematics, University of Michigan, Ann Arbor, MI}
\email{\href{mailto:saykhan@umich.edu}{saykhan@umich.edu}}
\urladdr{\url{https://www-personal.umich.edu/~saykhan/}}


\date{\today}

\begin{document}

\maketitle

\section{Corresponding results about $\mcg(\os_g)$}
\label{sec:corr-results-about}

\textbf{Referee comment:} It would be helpful if the author can include corresponding results about $\mcg(\os_g)$ in the 2\textsuperscript{nd} chapter as comparison.

\textbf{Edits:} Added Theorem 2.6 and Corollary 2.8 to illustrate what happens in the case of orientable surfaces.

\section{Different notions of limit set}
\label{sec:diff-noti-limit}

\section{Definition of $p_3$ in Section 4}
\label{sec:defin-p_3-sect}

\textbf{Referee comment:} What is the definition of $p_3$ in Chapter 4? Page 16, last paragraph, which means its length should be going to $\infty$, why? Your formula is a bit weird, the left side is $p_4$?

\textbf{Edits:} The curve $p_3$ is now defined after the proof of Claim 4.4. The claim itself shows that $p_3$ is well-defined.

\end{document}