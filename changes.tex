\documentclass[12pt, reqno]{amsart}
\pdfoutput=1
\usepackage{mathpreamble}

\title{Changes to ``The limit set of non-orientable mapping class groups''}
\author{Sayantan Khan}
\address{Department of Mathematics, University of Michigan, Ann Arbor, MI}
\email{\href{mailto:saykhan@umich.edu}{saykhan@umich.edu}}
\urladdr{\url{https://www-personal.umich.edu/~saykhan/}}


\date{\today}

\begin{document}

\maketitle

\section{Corresponding results about $\mcg(\os_g)$}
\label{sec:corr-results-about}

\textbf{Referee comment:} It would be helpful if the author can include corresponding results about $\mcg(\os_g)$ in the 2\textsuperscript{nd} chapter as comparison.

\textbf{Edits/Response:} Added Theorem 2.6 and Corollary 2.8 to illustrate what happens in the case of orientable surfaces.

\section{Different notions of limit set}
\label{sec:diff-noti-limit}

\textbf{Referee comment:} You have two different notions of limit set, could you state more precisely what did Erlandsson, Gendulphe, Pasquinelli, and Suoto prove? According what you wrote, they prove $\Gamma_{\text{dyn}} = \Gamma_{\text{geom}} = \pmf^+(\no_g)$, right? Is it possible to improve your method to conclude the same?

\textbf{Edits/Response:} Added a sentence and paragraph in the corresponding section about what the four author paper proves, as well as why the methods in this paper cannot quite prove the full result.

In particular, from the results of McCarthy-Papadopoulos, it follows that $\Lambda_{dyn}(\Gamma) \subseteq \Lambda_{geo}(\Gamma)$.
Both my paper, and the four author paper show that $\Lambda_{geo}(\Gamma) \subseteq \pmf^+(\no_g)$, but the four author paper subsequently also shows that $\Lambda_{dyn}(\Gamma) = \pmf^+(\no_g)$, which proves the equality, whereas my paper only proves a weaker version, which is that $\Lambda_{dyn}(\Gamma)$ contains a large subset of $\pmf^+(\no_g)$, described in Theorem 3.3.

\section{Definition of $p_3$ in Section 4}
\label{sec:defin-p_3-sect}

\textbf{Referee comment:} What is the definition of $p_3$ in Chapter 4? Page 16, last paragraph, which means its length should be going to $\infty$, why? Your formula is a bit weird, the left side is $p_4$?

\textbf{Edits/Response:} The curve $p_3$ is now defined after the proof of Claim 4.4. The claim itself shows that $p_3$ is well-defined.
The claims 4.4 and 4.5 are required to prove inequality (11), which has been changed to be just an upper bound for $\ell(p_3)$, which is all we need. And $\ell(p_3)$ goes to $\infty$ because it intersects $p_0$, which is a component of the limiting foliation.

\section{Weird notation in chapter 3}
\label{sec:weird-notat-chapt}

\textbf{Referee comment:} The whole chapter 3 has weird notation with $\lhyp(p_i)$. It would make more sense to use $\ell_i$ to represent the hyperbolic length w.r.t the metric $m_i$.
Also please reorganize your computation to make it more logical. The final contradiction is also weird, what do you mean by this can't happen if $\lhyp(p_i)$ approaches $\infty$? I thought it should be zero by your assumption. Your equation (18) should be interpreted algebraically in a better way.

\textbf{Remark:} I think the referee meant weird notation in chapter 4, since $\lhyp$ shows up in that chapter.

\textbf{Edits/Response:} All the instances of $\lhyp(p_k)$ have been changed to $\ell_i(p_k)$, which denotes the length of the curve $p_k$ (where $k \in \left\{ 0, 1, 2, 3, 4 \right\}$) with respect to the metric $m_i$.
Also, two additional inequalities have been added that clarify the estimates in the section. The estimate involves first showing that $\ell_i(p_1)$ goes to $\infty$, and the using that fact to show that $\frac{\ell_i(p_1)}{\ell_i(p_3)}$ cannot go to $0$, violating the assumptions made in case I.
Equation (17) was added to make equation (19) (which was previously (18)) easier to understand.

\section{Margulis lemma type result}
\label{sec:margulis-lemma-type}

\textbf{Referee comment:} I understand that you use hyperbolic geometry estimates to argue, but if you can, please have a more intuitive explanation. Is it similar to Margulis lemma type result?

\textbf{Edits/Response:} I added a section between the statement and proof of Proposition 4.3 that sketches out the two steps of the proof.
With regards to the Margulis lemma type result: I'm not quite sure I understand how the Margulis lemma comes in here. To my knowledge, that's a result about subgroups of Lie groups generated by elements in the neighbourhood of identity.
This result should be thought of as a converse to the collar lemma.
The collar lemma claims that a curve intersecting a short curve must be long.
In this result, we have a long curve (of a specific type) intersecting a one-sided curve, and we conclude that the one-sided curve must be short.

\section{Lower genera construction}
\label{sec:lower-genera-constr}

\textbf{Referee comment:} Why do you not have construction for lower genera?

\textbf{Edits/Response:} The construction actually works for all genera except genus $5$ and $7$.
To construct the counterexample in odd genus, I needed to add enough features on the DQD to satisfy the constraints of the construction, and that resulted in the genus going up to $9$ from $5$.
I am fairly certain that a similar construction will prove the result for genus $5$ and $7$, but that construction will possibly be more complicated to deal with the lack of ``room'' in the lower genus, and I chose to not follow that up since it would obscure the main idea of the proof.

I added a remark after the statement of Theorem 5.2 to this effect.

\section{Counting problem for $\no_g$}
\label{sec:counting-problem}

\textbf{Referee comment:} Could you state more precisely how does your result (or related result about limit set) help with the counting problem for $\no_g$?

\textbf{Edits/Response:} Added a few sentences in the ``Counting simple closed curves'' subsection citing Gamburd, Magee, and Ronan's result where they count simple closed curves on $\no_{1,3}$ using a conformal measure on the limit set.

\section{Link between Theorem 5.2 and limit set}
\label{sec:link-between-theorem}

\textbf{Referee comment:} In the introduction, you said on page 2, that the key idea involved in Theorem 5.2 is to determine the limit set; where do you use this in the proof? If not, can you find a different bridge between these two results.

\textbf{Edits/Response:} Added paragraphs after the statements of Theorem 5.2 and Theorem 3.3 in the introduction clarifying the link between the two results. While Theorem 3.3 is not directly used in the proof of Theorem 5.2, the description of the limit set provides a natural family of candidate geodesic segments, which Theorem 5.2 claims will do the job.

\end{document}